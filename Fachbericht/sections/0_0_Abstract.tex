\begin{abstract}
This semester’s assignment was to design and implement a control-unit for a commercial 3D-Printer. The requirements, apart from proper functioning were that the printer should be able to operate independently from a PC, have wireless connectivity and be able to read and print from G-Code files. Although not explicitly required, further enhancements were highly encouraged. The approach adopted was to reverse-engineer available open-source solutions adding enhancements along the process. Due to prior experience of the team with it, the ATmega2560 was the chosen microcontroller for this project. All the circuit elements were integrated in a single PCB, including an ESP8266 wireless module to provide the printer with WLAN connectivity. The 24 V power supply circuit and all the connectors were designed to be compatible with a Creality3D Ender-3 Pro 3D printer model. The freeware Marlin was the most suitable software due to its compatibility with a broad range of hardware components. Marlin’s code has been modified to provided additional features and a customized visual identity to the printer. With additional sensors the control-unit provides the 3D-printer filament runout detection, filament blockage detection and sensorless homing. After manufacturing the PCB and flashing Marlin into the microcontroller a battery of tests validated the functionality of the control-unit. However, the filament blockage detection function could not have its sensitivity adjusted to a satisfactory degree and should, therefore, not be used. The end-product is a fully functional and enhanced control-unit for an Ender-3 Pro printer, which can be made available to students of this school as a support tool for their projects. 
\end{abstract}	

\todo[inline]{Beispiel Text aus P4. Wird neu geschrieben.}