\clearpage

\section{Übersicht}\label{sec:Uebersicht}

\subsection{Ausgangslage und Aufgabenstellung}\label{subsec:AusgangslageundAufgabenstellung}

Im 2.4GHz ISM-Band kon­kur­ren­zie­ren sich derzeit die drei weit verbreiteten Low Power Mesh Netzwerk Protokolle Bluetooth Mesh, Thread und Zigbee.
Alle drei wurden für die kabellose Übertragung in sogenannten WSN (Wireless Sensor Networks) und für Smart Home Anwendungen konzipiert.
Während Thread und Zigbee den \textit{IEEE 802.15.4} Standard als Physical Layer benutzen, basiert der Bluetooth Mesh Stack auf dem \textit{Bluetooth Low Energy (BLE)} Standard.
Aufgrund der hohen Dichte an Protokollen und Systemen, die das 2.4GHz ISM-Band ebenso benutzen (z.B. Wifi), sind die Störeinflüsse auf die drei Mesh Protokolle nicht zu unterschätzen.
Die Protokollstacks begegnen diesem und weiteren Problemen auf unterschiedliche Weise.
Diese Unterschiede werden in dieser Arbeit aufgezeigt indem die Performance der Mesh Netzwerke unter unterschiedlichen Bedingungen getestet werden.
So soll ein qualitativer Vergleich der drei Mesh Protokolle gemacht werden.
Eine subjektive Bewertung der Mesh Stacks bezüglich deren Handhabung und Komplexität wird zudem vorgenommen.
Die detaillierte Ausgangslage und Aufgabenstellung kann einerseits der Aufgabenstellung im Anhang \ref{app:Aufgabenstellung} und andererseits dem Pflichtenheft im Anhang \ref{app:Pflichtenheft} entnommen werden.

\subsection{Vorarbeiten P5}\label{subsec:VorarbeitenP5}
Im Rahmen des Projekt 5 mit dem Titel \textit{Bluetooth-Mesh Plattform für IoT Anwendungen} wurde das Bluetooth Mesh Protokoll bereits vertieft betrachtet und dessen Vor- und Nachteile aufgezeigt. 
Basierend auf diesen Erkenntnissen und Erfahrungen sowie der oben beschriebenen Thematik wird in dieser Bachelor Thesis das Bluetooth-Mesh Protokoll mit den Alternativen Thread sowie Zigbee verglichen.
Die Ergebnisse des P5 haben gezeigt, dass sogenannte Flooding Mesh Netze, wie zum Beispiel Bluetooth Mesh, einige Nachteile gegenüber gerouteten Mesh Netzen aufweisen. Ob und wie diese ins Gewicht fallen und ob Bluetooth Mesh auch Vorteile gegenüber Thread und Zigbee hat, konnte im Projekt 5 nicht gezeigt werden und ist nun ein wichtiger Bestandteil dieser Thesis.

\subsection{Ziel der Arbeit}\label{subsec:ZielderArbeit}
Hauptziel dieser Thesis ist ein objektiver Vergleich der drei gängigsten Low Power Mesh Netzwerke Bluetooth Mesh, Thread und Zigbee bezüglich deren Leistungsfähigkeit unter wechselnden Bedingungen.
Es soll erkennbar werden, welches Protokoll in welchen Bereichen seine Stärken hat und wie es am besten eingesetzt werden kann.
Weitere subjektive Aspekte wie beispielsweise die Komplexität des Stacks oder die Interoperabilität mit anderen System sollen den Vergleich noch aussagekräftiger machen und die Praxistauglichkeit der Systeme aufzeigen.

Ein weiteres eigenständiges Ziel dieser Thesis ist, ein Instrument zu schaffen, um Signalmessungen auf den erwähnten Low Level Radio Treiber Ebenen \textit{IEEE 802.15.4} und \textit{BLE} durchführen zu können. Damit sollen beispielsweise geeignete Standorte für die Mesh Knoten ermittelt und ausserdem gezielte Störungen in die Mesh Benchmarks eingebracht werden können.
Die ausführliche Zielesetzung dieser Arbeit ist im Pflichtenheft im Anhang \ref{app:Pflichtenheft} aufgeführt.

\subsection{Abgrenzung}\label{sec:Abgrenzung}
Die vorliegende Thesis beschränkt sich auf die in den nachfolgenden Abschnitten beschriebenen Testumgebungen und Testkriterien. Weitere Parameter können in diesem Rahmen nicht betrachtet werden.
Daher müssen folgende Abgrenzungen beachtet werden:

Der Performance Vergleich der Mesh Netzwerke soll unter möglichst realen Bedingungen durchgeführt werden. Dies schliesst in gewisser Hinsicht auch sogenannte Stresstests, bei welchen die Mesh Stacks an ihre Leistungsgrenzen gebracht werden sollen, ein. Diese sollen jedoch in der vorliegenden Arbeit nicht im Zentrum stehen.

Weiter sollen die Mesh Benchmarks nachvollziehbar und reproduzierbar sein. Da je nach Umsetzung und Implementation des jeweiligen Mesh Stacks jedoch grosse Unterschiede in der Performance entstehen können, sind die Resultate nur im Bezug auf unsere Umsetzung zu interpretieren. Messungen mit divergierenden Stack Implementationen können von den Resultaten in dieser Arbeit abweichen.

