\clearpage

\section{Übersicht}\label{sec:Uebersicht}

\subsection{Ausgangslage und Aufgabenstellung}\label{subsec:AusgangslageundAufgabenstellung}
\todo[inline]{Cyrill}

\todo[inline]{Die Aufgabenstellung soll kurz zusammengefasst werden (komplette Aufgabenstellung im Anhang) und die Ausgangslage abgegrenzt werden.}

Im 2.4GHz ISM-Band kon­kur­ren­zie­ren sich derzeit die drei weit verbreiteten Low Power Mesh Netzwerk Protokolle Bluetooth Mesh, Thread und Zigbee.
Alle drei wurden konzipiert für die kabellose Übertragung in sogenannten WSN (Wireless Sensor Networks) oder in Netzen für die Heim Automatisierung. Während Thread und Zigbee den IEEE 802.15.4 Standard als Physical Layer benutzen basiert der BT Mesh Stack auf dem BLE (Bluetooth Low Energy) Standard. Aufgrund der hohen Dichte an Netzwerkprotokollen die das 2.4GHz ISM-Band ebenso nutzen (z.B. Wifi) sind die Störeinflüsse auf die Mesh Protokolle eines der grössten Probleme. Die Protokollstacks begegnen diesem und weiteren Problemen auf unterschiedliche Weise. Diese Unterschiede und schliesslich die Performance der Mesh Netzwerke sollen unter unterschiedlichen Testbedingungen aufgezeigt werden wodurch ein objektiver Vergleich der drei Mesh Protokolle möglich wird. Eine subjektive Bewertung der Mesh Stacks bezüglich deren Handhabung und Komplexität ist ausserdem wünschenswert.
Die detaillierte Ausgangslage und Aufgabenstellung kann einerseits der Aufgabenstellung im Anhang \ref{app:Aufgabenstellung} und andererseits dem Pflichtenheft im Anhang \ref{app:Pflichtenheft} entnommen werden.

\subsection{Vorarbeiten P5}\label{subsec:VorarbeitenP5}

Im Rahmen des P5 mit dem Namen Bluetooth-Mesh Plattform für IoT Anwendungen, wurde das Bluetooth-Mesh Protokoll bereits vertieft betrachtet und dessen Vor- und Nachteile aufgezeigt. 
Basierend auf diesen Erkenntnissen und Erfahrungen und der oben beschriebenen Thematik soll das Bluetooth-Mesh Protokoll mit den Alternativen Thread sowie Zigbee verglichen werden.
Die Ergebnisse des P5 haben gezeigt, dass sogenannte Flooding Mesh Netze wie es BT Mesh eines ist, gewisse Nachteile gegenüber gerouteten Mesh Netzen aufweisen. Ob und wie diese ins Gewicht fallen und ob BT Mesh auch Vorteile gegenüber Thread und Zigbee hat konnte im P5 nicht gezeigt werden und ist nun Teil dieser Thesis.

\subsection{Ziel der Arbeit}\label{subsec:ZielderArbeit}

Das Hauptziel dieser Thesis ist es den König unter den Low Power Mesh Netzwerken zu krönen. Dies mit dem Bewusstsein, dass es diesen einen Sieger wohl nicht geben wird. Viel eher sollen Disziplinen Sieger gekürt werden und somit aufgezeigt werden in welchen Situationen welches Mesh Protokoll am Leistungsfähigsten ist und welches am besten praxistauglich ist.
Weiter soll ein Instrument geschaffen werden um Signalmessungen auf den erwähnten Low Level Radio Treiber Ebenen IEEE 802.15.4 und BLE durchführen zu können. Damit sollen beispielsweise geeignete Standorte für die Mesh Knoten ermittelt und ausserdem gezielte Störungen in die Mesh Benchmarks eingebracht werden können.
Die ausführlichen Ziele dieser Arbeit sind im Pflichtenheft im Anhang \ref{app:Pflichtenheft} aufgeführt.


\subsection{Abgrenzung}\label{sec:Abgrenzung}

Folgende Abgrenzungen müssen bei der vorliegenden Arbeit beachtet werden.

Der Performance Vergleich der Mesh Netzwerke soll unter möglichst realen Bedingungen durchgeführt werden. Dies schliesst zwar in gewissem Mass auch sogenannte Stresstests mit ein bei welchen die Mesh Stacks an ihre Grenzen gebracht werden sollen, diese sollen jedoch nicht im Zentrum stehen.

Weiter sollen die Mesh Benchmarks nachvollziehbar und reproduzierbar sein. Da je nach Umsetzung und Implementation des jeweiligen Mesh Stacks jedoch grosse Unterschiede in der Performance entstehen können sind die Resultate nur im Bezug auf unsere Umsetzung zu interpretieren. Messungen mit divergierende Stack Implementationen können von unseren Resultaten abweichen.

