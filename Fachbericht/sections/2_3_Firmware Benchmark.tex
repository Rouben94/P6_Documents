\clearpage
\section{Umsetzung Benchmark}\label{sec:ThreadUmsetzungBenchmark}
In diesem Kapitel wird die Thread Stack spezifische Benchmark Umsetzung beschrieben. Das gesamte Projekt ist im \href{https://github.com/Rouben94/P6_Software}{Github-Repository\footnote{\url{https://github.com/Rouben94/P6_Software}\cite{github_p6_software_p2p_2020}}} unter dem Ordner \textbf{P6\_Software/Openthread/ Thread\_SDK/custom/P6\_Project/ot\_benchmark/} verfügbar. Ausserdem sind im Ordner \textbf{P6\_Software/Openthread/Thread\_SDK/custom/benchmark\_hex/} die vorkompilierten .zip Dateien, um diese auf den nRF52840 SoC zu flashen. 

\subsection{Openthread Stack Konfiguration}\label{subsec:ThreadStackKonfiguration}
\todo[inline]{Konfig erklären}
\begin{table}[h]
	\centering
	\begin{adjustbox}{width=1\textwidth}
		\begin{tabular}{lll}
			Stack Init Time                            & 30000 ms                                                               & \begin{tabular}[c]{@{}l@{}}Zeit die benötigt wird um den Stack für den Benchmark\\zu initialisieren. Reserve Zeit um das Routing des \\Netzwerkes durchzuführen.\end{tabular}  \\
			IEEE Channel                               & 15                                                                     & Thread Kanal der verwendet wird um das Mesh aufzubauen.                                                                                                                        \\
			Netzwerk Name                              & OpenThread                                                             & Netzwerkname~                                                                                                                                                                  \\
			\textcolor[rgb]{0.278,0.278,0.278}{PAN ID} & \textcolor[rgb]{0.278,0.278,0.278}{0xABCD}                             & Wird für das Precomissioning verwendet.                                                                                                                                        \\
			Netzwerkschlüssel                          & \textcolor[rgb]{0.278,0.278,0.278}{0x00112233445566778899AABBCCDDEEFF} & Wird für das Precomissioning verwendet.                                                                                                                                       
		\end{tabular}
	\end{adjustbox}
	\caption{Messgrössen Gewinnung und Verwendung}
	\label{tab:ThreadMessgrössenGewinnungundVerwendung}
\end{table}

\newpage
\subsection{CoAP Benchmark Message}\label{subsec:ThreadBenchmarkMessage}
Um die Benchmark Message vom Client an den Server zu senden wurde ein CoAP-Protokoll implementiert. Das Protokoll ist im Kapitel \ref{subsubsec:CoAP} näher erläutert. Folgend wird beschrieben, welche Funktionen implementiert wurden.

\paragraph{Default Handler} 
Der Default-Handler ist einer Callback-Funktion, die jede CoAP Applikation beinhalten muss. Bei der Initialisierung von CoAP muss eine Callbackfunktion mitgegeben werden. Diese wird aufgerufen, falls die CoAP Nachricht nicht zugeordnet werden kann. Das bedeutet die Adresse der Nachricht stimmt, jedoch ist der Uniform Resource Identifiert (URI) der Nachricht nicht registriert. Der URI wird verwendet, um die Nachricht zu identifizieren. 

\paragraph{Probe Message Send}
Nur Clientseitig implementiert ist die Funktion \textit{bm\_coap\_probe\_message\_send}. Mit dieser Funktion wird eine neue Nachricht generiert und an die entsprechende Multicast Adresse gesendet. Zusätzlich zu der Adresse wird ein CoAP Port benötigt. Dieser ist standartmässig auf die Portadresse \textit{0x1633} festgelegt. Folgende Parameter werden der Nachricht mit der gesendet:

\begin{itemize}
	\item \textbf{URI:} bm\_probe
	\item \textbf{Message ID:} Die aktuelle Message ID, um die Nachricht identifizieren zu können. (1 byte)
	\item \textbf{Source Adress:} Die Adresse des Clients, damit in der Auswertung die Beziehungen ausgewertet werden können. (2 byte)
	\item \textbf{Payload :} Zusätzliche Payload, die mit zufälligen Daten gefüllt wird (1 - 1021 byte)
\end{itemize}

Bevor die Nachricht gesendet wird, werden folgende Parameter ausgelesen und ins RAM gespeichert:

\begin{table}[H]
	\centering
	\begin{tabular}{|l|l|}
		\hline
		\textbf{Parameter}  & \textbf{Erklärung}                                                                                               \\ \hline
		Source Address      & EID Adresse vom Client                                                                                           \\ \hline
		Destination Address & Ziel Adresse an den die Nachricht gesendet wird.                                                                 \\ \hline
		Group Address       & Aktuelle Gruppen Adresse, die initialisiert ist.                                                                 \\ \hline
		Message ID          & Aktuelle Message ID                                                                                              \\ \hline
		Network Time        & \begin{tabular}[c]{@{}l@{}}Aktuelle Netzwerk Zeit, die mit der Zeitsynchronisation\\ verfügbar ist.\end{tabular} \\ \hline
		Payload Length      & Zusätzlich Payload, die mitgegeben werden kann.                                                                  \\ \hline
	\end{tabular}
	\caption{Messparameter Client Thread}
	\label{table:MessparameterClientThread}
\end{table}

\newpage
\paragraph{Probe Message Handler}
Um die Benchmark Message auf dem Server zu erhalten, wird eine Callbackfunktion mit dem entsprechenden URI (\textit{bm\_probe}) initialisiert. Wird die Funktion \textit{bm\_probe\_message\_handler} nach eingehen einer validen CoAP Nachricht aufgerufen, so werden folgender Parameter entweder aus der Nachricht ausgelesen oder vom System abgerufen und ins RAM gespeichert:

\begin{table}[H]
	\centering
	\begin{adjustbox}{width=1\textwidth}
		\begin{tabular}{|l|l|l|}
			\hline
			\textbf{Parameter}  & \textbf{Erklärung}                                                                                               & \textbf{In der Payload mitgegeben} \\ \hline
			Source Address      & EID Adresse vom Client                                                                                           & \multicolumn{1}{c|}{X}             \\ \hline
			Destination Address & EID Adresse vom Server                                                                                           &                                    \\ \hline
			Group Address       & Aktuelle Gruppen Adresse, die initialisiert ist.                                                                 &                                    \\ \hline
			Message ID          & Aktuelle Message ID                                                                                              & \multicolumn{1}{c|}{X}             \\ \hline
			Network Time        & \begin{tabular}[c]{@{}l@{}}Aktuelle Netzwerk Zeit, die mit der Zeitsynchronisation\\ verfügbar ist.\end{tabular} &                                    \\ \hline
			Payload Length      & Zusätzlich Payload, die mitgegeben werden kann.                                                                  & \multicolumn{1}{c|}{}              \\ \hline
			RSSI                & RSSI Durchschnittswert über alle Hops                                                                            &                                    \\ \hline
			Number od Hops      & Anzahl Hops die die Nachricht genommen hat                                                                       &                                    \\ \hline
		\end{tabular}
	\end{adjustbox}
	\caption{Messparameter Server Thread}
	\label{table:MessparameterServerThread}
\end{table}

\subsection{IPv6 Adressierung}\label{subsec:IPv6Adressierung}
Die einzelnen Knoten im Benchmark Test sollten in Gruppen konfiguriert werden, da dies der Praxis einer Heimautomation entspricht. Dort können verschiedene Schalter mehrere Leuchten an- und ausschalten. Somit bietet sich die Multicast Adressierung von Thread an, da mehrere Nachricht gleichzeitig versendet werden können wie im Paragraph \ref{par:ThreadMulticast} beschrieben. Um diese Gruppenadressierung mit Multicast Adressen umzusetzen, wurden 25 verschiedene Adressen vordefiniert:

\begin{table}[H]
	\centering
	\begin{tabular}{|l|l|}
		\hline
		\textbf{Gruppen Nummber} & \textbf{Multicast Adresse} \\ \hline
		1                        & ff03::3                    \\ \hline
		2                        & ff03::4                    \\ \hline
		3                        & ff03::5                    \\ \hline
		4                        & ff03::6                    \\ \hline
		...                      & ...                        \\ \hline
		25                       & ff03::27                   \\ \hline
	\end{tabular}
	\caption{Gruppenadressierung Thread}
	\label{table:GruppenadressierungThread}
\end{table}