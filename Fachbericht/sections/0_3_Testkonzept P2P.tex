\clearpage
\section{Point to Point Testinfrastruktur}\label{sec:PointtoPointTestinfrastruktur}

Im Nachfolgenden Abschnitt wird auf die Point to Point (P2P) Messung eingegangen. Ziel ist es die Verbindung auf MAC-Layer auszumessen und die aus Abschnitt ... aufgelisteten Messdaten zu erfassen. Dazu wird im folgenden auf das Messkonzept, den Messablauf und technische Einzelheiten eingegangen. 

\subsection{Konzeptschema}\label{sec:KonzeptschemaP2P}
\todo[inline]{Erläuterung des Konzeptschemas}

\begin{figure} [H]
	\centering
	\includegraphics[width=0.8\textwidth]{Konzeptschema_P2P.png}
	\caption{Konzeptschema P2P}
	\label{fig:KonzeptschemaP2P}
\end{figure}

Das in Abbildung \ref{fig:KonzeptschemaP2P} gezeigte Konzept besteht aus einer Managment Station, einem Master- und mehreren Slavesnodes. Über die Management Station können Messresultate angezeigt, sowie Messparameter eingestellt werden.  

\subsection{Testszenarien}\label{sec:TestszenarienP2P}
\todo[inline]{Testumgebungen sowie die Beziehungen der Knoten innerhalb der Point to Point Messung beschreiben.}

\subsection{Ablauf}\label{sec:AblaufP2P}
\todo[inline]{Ablauf eines Point to Point MAC Layer Benchmarks aus Anwendersicht beschreiben.}

\subsection{Messgrössen}\label{sec:MessgrössenP2P}
\todo[inline]{Erläuterung der Messgrössen die erfasst werden sollen. Inkl. Beschreibung wie dies technisch umgesetzt wird.}

\subsection{Messaufbau}\label{sec:Messaufbau}
\todo[inline]{Schema inkl. Beschreibung.}

\subsection{Aufbau und Bedienung der Messinfrastruktur}\label{sec:AufbauundBedienungderMessinfrastruktur}
\todo[inline]{Bedienungsanleitung des Messaufbaus.}

\subsection{Interpretation der Messresultate für den Anwender}\label{sec:InterpretationderMessresultatefürdenAnwender}
\todo[inline]{Wie sind die Resultate zu interpretieren. Welche Schlüsse können/müssen aus den Resultaten gezogen werden. Welche Resultate bedeuten was?}

\subsection{Soft- und Firmware}\label{sec:SoftundFirmware}
\todo[inline]{Beschreibung der Node-Firmware sowie der Auswertesoftware auf dem Raspi.}

\subsection{Zeitsynchronisation}\label{sec:Zeitsynchronisation}
\todo[inline]{Wie wurde die Zeitsynchronisation umgesetzt? Wie relevant ist diese für die Messungen?}