\vspace*{4cm}
\part{Zigbee}\label{part:Zigbee}
Cyrill Horath
\vspace*{\fill}
\clearpage

\section{Einleitung}\label{sec:EinleitungZigbee}
\todo[inline]{Kurze Einleitung ins Thema Zigbee. Dieser Teil ist komplett losgelöst von den anderen Teilen. Es soll klar ersichtlich sein dass dieser Teil durch Cyrill Horath erstellt wurde.}

Zigbee ist ein auf dem IEEE 802.15.4 Standard aufbauendes drahtloses Low Power Mesh Netzwerk. Es nutzt das vom IEEE 802.15.4 Standard definierte ISM-Funkfrequenzband 2.4GHz plus weitere Sub-GHz Bänder je nach Region.
Die im Jahre 2002 gegründete Zigbee Allianz spezifiziert den Protokoll Standard ist gibt seit da an laufend Neuerungen und Updates heraus.
Im Zuge der Verbreitung von Home Automation Technologien erhielt auch Zigbee immer mehr Aufmerksamkeit und wuchs bis heute zu einem der am weitesten verbreiteten Protokolle für sogenannte WSN (Wireless Sensor Networks) und vorallem Lichtsteuerungssystem heran. Phillips Hue und Ikea Tradfri sind nur zwei Beispiele in welchen Zigbee verbreitet zum Einsatz kommt.





