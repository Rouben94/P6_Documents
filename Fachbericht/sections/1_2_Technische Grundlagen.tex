\clearpage
\section{Technische Grundlagen Bluetooth Mesh}\label{sec:TechnischeGrundlagenBluetoothMesh}




\subsection{Netzaufbau und Topologie}\label{sec:NetzaufbauundTopologie}

Nebst dem Funktionellen Aspekt übernehmen Nodes unterschiedliche Rollen im Netzaufbau. Ein Node wird als Relay-Node bezeichnet wenn dieser Nachrichten an weitere Teilnehmer weiterleitet. Ein Friend-Node dient als Zugangspunkt für einen Low-Power-Node. Der Low-Power-Node wird dort eingesetzt wo keine konstante Stromversorgung zur Verfügung steht. Dieser geht eine Beziehung mit einem Friend-Node ein. Der Friend-Node speichert alle Nachrichten der LPNs, welche mit ihm eine Beziehung pflegen. Über ein Zeitintervall fragt der LPN die verpassten Nachrichten beim Friend ab. Dadurch kann der LPN zwischen Abfragen inaktiv sein um Energie zu sparen. Um die Interoperabilität zwischen inkompatiblen Bluetooth-Mesh Geräten und einem Mesh-Netzwerk zu ermöglichen existieren Proxy-Nodes. Ein Proxy-Node dient als Schnittstelle in das Netzwerk und erlaubt das Interagieren über Bluetooth-GATT mit dem Mesh. Die in Abbildung \ref{fig:BTMeshTopology} gezeigt Topologie zeigt die verschiedenen Node-Typen an ihrem Einsatzort. 

\begin{figure} [H]
	\centering
	\includegraphics[width=1.0\textwidth]{Bluetooth_Mesh_Topology.PNG}
	\caption{Topologie eines Bluetooth-Mesh Netzwerks \cite{bluetooth_sig_mesh_netzwerk_spezifikationen_2020}} 
	\label{fig:BTMeshTopology}
\end{figure}





\todo[inline]{Welchen Aufbau? Welche Art von Mesh? Welche Nodetypen gibt es? Welche typischen Eigenschaften besitzt das Protokoll?}

\subsection{Bluetooth Mesh Protokoll Stack}\label{sec:BLEMeshProtokollStack}

In diesem Abschnitt wird die Architektur des Mesh-Stacks genauer untersucht. Wie in Kapitel \ref{sec:EinleitungBluetooth} bereits erwähnt basiert der Stack auf Bluetooth Low Energy. Der BLE-Layer dient zur Grundlegenden Schicht des Stacks. Der Zugriff für Mesh-Traffic erfolgt über das GAP-Profil, der für Proxy Traffic über das GATT-Profil. Der Bearer-Layer regelt den Zugriff auf den BLE-Stack. Es existieren verschiedene Bearers. Der GATT-Bearer ermöglicht Geräten ohne GAP Zugriff auf das Netzwerk. Der Advertising-Bearer wird für den Mesh-Traffic benutzt. \\

Der Network-Layer hat diverse Aufgaben zu erfüllen: 

\begin{itemize}
	\item Ver- und Entschlüsselung der Network-PDU
	\item Filtern von nicht relevanten Nachrichten (Adressauflösung)
	\item Relaying von Paketen mittels TTL-Field
\end{itemize}

Zudem bedient dieser Layer verschiedene Bearers und ist dafür verantwortlich das alle relevanten Pakete zur entsprechenden Stelle weitergeleitet werden. 


\begin{figure} [H]
	\centering
	\includegraphics[width=0.5\textwidth]{Bluetooth_Mesh_Stack_Layers.PNG}
	\caption{Bluetooth-Mesh Stack \cite{bluetooth_sig_mesh-technology-overviewpdf_2020}} 
	\label{fig:BTMeshStack}
\end{figure}

\begin{figure} [H]
	\centering
	\includegraphics[width=0.9\textwidth]{BLE_MESH_Message_Flow_Diagramm_UpperLayers.PNG}
	\caption{Bluetooth-Mesh Stack Upper Layers \cite{bluetooth_sig_mesh_netzwerk_spezifikationen_2020}} 
	\label{fig:BTMeshStackUpperLayers}
\end{figure}

\begin{figure} [H]
	\centering
	\includegraphics[width=1.0\textwidth]{BLE_MESH_Message_Flow_Diagramm_LowerLayers.PNG}
	\caption{Bluetooth-Mesh Stack Lower Layers \cite{bluetooth_sig_mesh_netzwerk_spezifikationen_2020}} 
	\label{fig:BTMeshStackLowerLayers}
\end{figure}

\todo[inline]{Erläuterung des Protokoll Stacks. Möglichst viel Grafiken und nur so viel als nötig Prosa.}

\subsection{Bluetooth Mesh Software Development Kit}\label{sec:ZigbeeSoftwareDevelopmentKit}
\todo[inline]{Eingesetzte SDK und deren Aufbau beschreiben. Allenfalls die wichtigsten API Funktionen genauer erläutern.}


