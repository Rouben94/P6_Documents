\clearpage
\section{Benchmark-Konzept Mesh Netzwerke}\label{sec:BenchmarkKonzeptMeshNetzwerke}
\todo[inline]{Cyrill}

\subsection{Konzeptschema}\label{subsec:KonzeptschemaMesh}
\todo[inline]{Erläuterung des Konzeptschemas}

\subsubsection{Nodes}\label{subsubsec:Nodes}

\subsubsection{Messages}\label{subsubsec:Messages}


\subsection{Testszenarien}\label{subsec:TestszenarienMesh}
\todo[inline]{Testumgebungen sowie die Beziehungen der Knoten innerhalb der Mesh Netze beschreiben.}

\subsection{Ablauf}\label{subsec:AblaufMesh}
\todo[inline]{Ablauf eines Mesh Benchmarks aus Anwendersicht beschreiben.}


\subsection{Messaufbau}\label{subsec:Messaufbau}
\todo[inline]{Wenn möglich Schema mit dem Aufbau der Messumgebung. Beschreibung der unterschiedlichen Standorte.}

\subsection{Messgrössen}\label{subsec:MessgrössenMesh}
\todo[inline]{Erläuterung der Messgrössen die erfasst werden sollen. Inkl. Beschreibung wie dies technisch umgesetzt wird.}

\subsection{Messdatenerfassung und Auswertung}\label{subsec:MessdatenerfassungundAuswertung}
\todo[inline]{Beschreibung der Hard und Software für die Datenerfassung und die Auswertung.}

\subsection{Messerwartung}\label{subsec:Messerwartung}
\todo[inline]{Welche Resultate werden erwartet. Welcher der drei Stacks ist vermeintlich der Beste?}

