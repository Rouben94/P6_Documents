\vspace*{4cm}
\part{Bluetooth Mesh}\label{part:BluetoothMesh}
Raffael Anklin
\vspace*{\fill}
\clearpage

\section{Einleitung}\label{sec:EinleitungBluetooth}


Bluetooth Mesh ist ein auf dem Bluetooth-Standard aufbauendes Mesh-Netzwerk. Der Standard wurde im Jahr 2017 von der Bluetooth-SIG vorgestellt. Die Technologie baut auf den weit verbreitetem BLE-Standard auf, welcher in einer viel zahl von Endgeräten zur Kommunikation genutzt wird. Alle Geräte ab BLE-Version 4.2 sind Kompatibel und können in ein Mesh-Netzwerk eingebunden werden. Organisiert wird das Netzwerk nicht durch ein Routing, sondern basiert auf einem Managed-Flooding Prinzip. Einfach erklärt wiederholen alle Teilnehmer, bedingt durch verschiedene Abhängigkeiten, jede empfangende Nachricht (Relaying). Somit gelangen die Daten über Zwischenstationen (Hops) zum Ziel. 

\todo[inline]{Bild Managed Flooding}

\todo[inline]{Kurze Einleitung ins Thema Bluetooth Mesh. Dieser Teil ist komplett losgelöst von den anderen Teilen. Es soll klar ersichtlich sein dass dieser Teil durch Raffael Anklin erstellt wurde.}







