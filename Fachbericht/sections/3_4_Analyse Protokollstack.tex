\clearpage
\section{Analyse Zigbee Stack}\label{sec:AnalyseZigbeeStack}
Für die Umsetzung des Benchmark Konzepts war eine vertiefte Auseinandersetzung mit dem Zigbee Protokoll nötig.
Dabei konnten grosse Erfahrungen mit dem Stack sowie der verwendeten SDK gesammelt werden.
Aufgrund dieser Erfahrungen wird nachfolgend eine Analyse des Zigbee Protokollstacks sowie der eingesetzten SDK gemacht und einige Schwierigkeiten und Hindernisse bei der Softwareentwicklung aufgezeigt.

\subsection{Stärken und Schwächen des Protokollstacks}\label{subsec:ZigbeeStärkenundSchwächendesProtokollstacks}
Folgende Stärken und Schwächen weist der \textit{ZBOSS Zigbee Stack} auf.
Stack-Implementationen von anderen Herstellern können nicht bewertet werden.

\textbf{Stärken:}
\begin{itemize}
\item Die \textbf{Zigbee Cluster Library (ZCL)} erleichtert erheblich die Umsetzung eines Produktes, welches mit anderen Zigbee Geräten konform sein soll.
\item Die \textbf{Adressierung mit Short-, Long- und Group-Adressen} ist übersichtlich und schlank gehalten.
Dies vereinfacht das Handling für Entwickler und Anwender.
\item Ein einfacher \textbf{Scheduler und Callback} Mechanismus vereinfacht das Nachrichten Handling.
\item Der \textbf{IEEE 802.15.4 Standard} für die MAC Ebene bietet dem Entwickler eine solide Basis, um die gewünschte Anwendung darauf umsetzen zu können.
\end{itemize}

\textbf{Schwächen:}
\begin{itemize}
\item Bei der \textbf{Gruppen-Adressierung} zeigt der Zigbee Stack deutliche Schwächen, da für Broadcast Nachrichten ein kleinerer Packetbuffer verwendet wird.
\item Wenn \textbf{Zigbee Cluster Library (ZCL)}-Funktionalitäten nicht zwingend benötigt werden, wird die Applikationsentwicklung umständlich.
Der Stack ist zu sehr auf vordefinierte Anwendungen ausgerichtet.
\item Die Spezifikationen des Stacks sind frei verfügbar. Leider implementieren die Hersteller den Stack schliesslich im \textbf{Closed-Source} Verfahren weshalb keine Community entstehen kann, die den Stack verbessert.
\item Das \textbf{Commissioning} von neuen Teilnehmern hat in unserer Anwendung leider nicht immer reibungslos funktioniert.
Bei vielen gleichzeitigen Anmeldeanfragen kam der \textbf{Zigbee-Koordinator} gerne ins Stocken und der Vorgang musste neu gestartet werden.
\end{itemize}


\subsection{Schwierigkeiten bei der Umsetzung}\label{subsec:ZigbeeSchwierigkeitenbeiderUmsetzung}
Einige Probleme haben die Entwicklung der Zigbee Benchmark Firmware schwieriger gemacht als erwartet.
Die folgenden fünf Punkte haben die Arbeiten merklich beeinflusst. 

\paragraph{Software Development Kit}
Da die \textit{nRF5 SDK for Thread and Zigbee} wohl bald von der \textit{nRF Connect SDK} abgelöst wird, scheint der Hersteller Nordic Semiconductor nicht wirklich Interesse daran zu haben, die \glqq alte\grqq{} SDK zu unterhalten.
Dies macht sich in der Dokumentation der SDK bemerkbar, denn gewisse API's sind nicht oder nur sehr schlecht dokumentiert.
Da es sich beim \textit{ZBOSS Zigbee Stack} um eine vorkompilierte Library handelt, ist auch die Interpretation des Codes schwierig, wenn nicht sogar unmöglich.
In einigen Fällen verweist die Dokumentation sogar auf Funktionen und Beispiele, die in der aktuellsten Version der SDK nicht mehr verfügbar sind.

\paragraph{Custom APS Data}
Die \textit{nRF5 SDK for Thread and Zigbee} bietet die Möglichkeit, sogenannte \textit{Custom APS Data} zu versenden.
Dabei handelt es sich um eine Low Level API für den Versand von benutzerdefinierten Anwendungsdaten, ohne dass die ZCL benutzt werden muss.
Während der Umsetzung des Benchmarks hat sich gezeigt, dass der Empfang und die Auswertung dieser Daten nur sehr schwierig umsetzbar sind.
Die entsprechende Dokumentation innerhalb der SDK war dazu nicht ausreichend detailliert.
Im Benchmark Kontext hätte diese Funktion den Versand simpler Applikationsdaten vereinfacht.

\paragraph{Anzahl Hops}
Für die Auswertung der Messergebnisse in den Abschnitten \ref{sec:Messresultate} und \ref{sec:VergleichMeshNetzwerke} wurden unterschiedliche Messgrössen erfasst.
Unter anderem wäre auch die Anzahl Hops, die eine Benchmark Nachricht innerhalb des Mesh Netzwerkes passiert hat, von grosser Wichtigkeit.
Nur so kann die Latenzzeit unabhängig vom gewählten Weg definiert und mit den Latenzzeiten der anderen Protokolle verglichen werden.
Leider bietet die \textit{nRF5 SDK for Thread and Zigbee} keine Möglichkeit, diesen Wert zu bestimmen, da das Radius-Feld aus dem NWK-Header nicht ausgelesen werden kann.
Das Radius-Feld wäre gemäss neuster \href{https://zigbeealliance.org/solution/zigbee}{Zigbee Specification\footnote{\url{https://zigbeealliance.org/solution/zigbee}\cite{the_zigbee_alliance_zigbee_2015}}} ein 8-Bit Feld im NWK-Header, welches beim passieren eines Hops dekrementiert wird \cite{the_zigbee_alliance_zigbee_2015}.
Das Paket wird verworfen, sobald das Feld den Wert 0 erreicht.
Der folgende Beitrag aus dem Supportforum von Nordic Semiconductor bestätigt dieses Problem: \href{https://devzone.nordicsemi.com/f/nordic-q-a/63815/zigbee---read-number-of-hops-radius}{Zigbee - Read number of hops (radius)\footnote{\url{https://devzone.nordicsemi.com/f/nordic-q-a/63815/zigbee---read-number-of-hops-radius}\cite{cyrill_horath_zigbee_2020}}}.

\paragraph{Group addressing mode}
Diverse Tests mit implementiertem \textit{ZCL groups cluster} haben gezeigt, dass Zigbee bei der Adressierung von Gruppenadressen schnell überfordert ist.
Selbst bei durchschnittlichem Verkehrsaufkommen im Netzwerk, wurde ein grosser Packetverlust von über 90 Prozent festgestellt.
Getestet wurde dies mit den Messreihen 1 und 2 gemäss Abschnitt \ref{subsec:Messreihe}.
Eine Gruppenadressierung löst in Zigbee eine Broadcast Nachricht aus, welche von allen Teilnehmern weitergeleitet wird.
Dies verursacht ein relativ grosses Verkehrsaufkommen im Netz. 
Es kann vermutet werden, dass der Paketbuffer für Broadcast Nachrichten zu klein ist und die Nachrichten zu langsam abgearbeitet werden.
Dadurch überläuft der Paketbuffer zu schnell und es entstehen die erwähnten Probleme.
Es konnte festgestellt werden, dass nach ca. 15 versendeten Nachrichten keine weiteren Nachrichten zugestellt werden können, bis sich der Stack nach einiger Zeit wieder erholt hat.

\paragraph{nRF Connect SDK}
Nordic Semiconductor stellt nebst der in dieser Arbeit verwendeten \textit{nRF5 SDK for Thread and Zigbee} noch eine weitere SDK für die Entwicklung von Zigbee Produkten zur Verfügung.
Die \textit{nRF Connect SDK} unterstützt seit der Version v1.3.0 ebenfalls den Zigbee Protokollstack.
Im Verlauf dieser Arbeit wurde auch diese SDK testweise eingesetzt.
Leider hat sich erst spät herausgestellt, dass die \textit{nRF Connect SDK} noch einige \glqq Kinderkrankheiten\grqq{} bei der Verwendung mit Zigbee besitzt.
So werden Nachrichten teilweise erst mit einer Verzögerung von 200ms versendet. In den \href{https://developer.nordicsemi.com/nRF_Connect_SDK/doc/1.3.0/nrf/doc/release-notes-1.3.0.html}{Release Notes\footnote{\url{https://developer.nordicsemi.com/nRF_Connect_SDK/doc/1.3.0/nrf/doc/release-notes-1.3.0.html}}} zu der entsprechenden Version ist dies festgehalten. 
