\clearpage
\section{Analyse Zigbee Stack}\label{sec:AnalyseZigbeeStack}
\todo[inline]{Persönliche Eindrücke, Vor- und Nachteile, Besonderheiten beschreiben. Soll dem Leser helfen die "Qualität" des Stacks einzuordnen.}

Für die Umsetzung des Benchmark Konzepts war eine vertiefte Auseinandersetzung mit dem Zigbee Protokoll nötig.
Dabei konnten einige Erfahrungen mit dem Stack sowie der verwendeten SDK gemacht werden.
Aufgrund dieser Erfahrungen wird nachfolgend eine Analyse des Zigbee Protokollstacks gemacht und einige Schwierigkeiten und Hindernisse bei der Softwareentwicklung aufgezeigt.


\subsection{Stärken und Schwächen des Protokollstacks}\label{subsec:ZigbeeStärkenundSchwächendesProtokollstacks}

\paragraph{Stärken}

\begin{itemize}
\item Die \textbf{Zigbee Cluster Library} erleichtert die Umsetzung eines Produktes, welches mit anderen Zigbee Geräten konform sein soll, erheblich.
\item 
\end{itemize}

\paragraph{Schwächen}
\begin{itemize}
\item Bei der \textbf{Gruppen Adressierung} zeigt der Zigbee Stack deutliche Schwächen da für Broadcast Nachrichten ein kleinerer Packetbuffer verwendet wird.
\end{itemize}


\subsection{Schwierigkeiten bei der Umsetzung}\label{subsec:ZigbeeSchwierigkeitenbeiderUmsetzung}

\subsubsection{Software Development Kit}\label{subsubsec:ZigbeeSoftwareDevelopmentKit}

\subsubsection{Anzahl Hops}\label{subsubsec:AnzahlHops}
Für die Auswertung der Messergebnisse in den Abschnitten \ref{sec:Messresultate} und \ref{sec:VergleichMeshNetzwerke} wurden unterschiedliche Messgrössen erfasst.
Unter anderem wäre auch die Anzahl Hops die eine Benchmark Nachricht innerhalb des Mesh Netzwerkes passiert hat, von grosser Wichtigkeit.
Nur so kann die Latenz unabhängig von gewählten Weg definiert und mit den Latenzzeiten der anderen Protokolle verglichen werden.
Leider bietet die \textit{nRF5 SDK for Thread and Zigbee} keine Möglichkeit diesen Wert zu bestimmen, da das Radius-Feld aus dem NWK-Header nicht ausgelesen werden kann.
Der folgende Beitrag aus dem Supportforum von Nordic Semiconductor bestätigt dies: \href{https://devzone.nordicsemi.com/f/nordic-q-a/63815/zigbee---read-number-of-hops-radius}{\textit{Zigbee - Read number of hops (radius)\footnote{\url{https://devzone.nordicsemi.com/f/nordic-q-a/63815/zigbee---read-number-of-hops-radius}\cite{cyrill_horath_zigbee_2020}}}}


\subsubsection{Group addressing mode}\label{subsubsec:Groupaddressingmode}
Diverse Tests mit implementiertem \textit{ZCL groups cluster} haben gezeigt, dass Zigbee bei der Adressierung von Gruppenadressen schnell überfordert ist.
Besonders bei leicht erhöhtem Verkehrsaufkommen im Netzwerk, konnte grosser Packetverlust von über 90 Prozent festgestellt werden.
Eine Gruppenadressierung löst in Zigbee eine Broadcast Nachricht aus, welche von allen Teilnehmern weitergeleitet wird.
Dies verursacht wiederum ein noch grösseres Verkehrsaufkommen. 
Grund für die Problematik ist vermutlich, dass der Packetbuffer für Broadcast Nachrichten zu klein ist sowie zu langsam abgearbeitet wird und daher schnell überläuft.
Es konnte festgestellt werden, dass nach ca. 16 Nachrichten, keine weiteren Nachrichten zugestellt werden konnten bis sich der Stack nach einiger Zeit wieder erholen konnte.

\subsubsection{nRF Connect SDK}\label{subsubsec:nRFConnectSDK}


\subsection{Erkenntnisse und Fazit}\label{subsec:ErkenntnisseundFazit}



\todo[inline]{Versuch mit nRF Connect SDK erwähnen.}
\todo[inline]{ZCL Vor- und Nachteile. Kompliziert mit Attributen Definition usw. Hilfreich da herstellerunabhängig}
\todo[inline]{SDK schlecht dokumentiert. ZBOSS Funktionen teilweise gar nicht dokumentiert oder alte Dokumentation von Funktionen die nicht mehr vorhanden sind.}
\todo[inline]{ZBOSS Scheduler hilfreich für das Handling in der Applikation.}


\todo[inline]{Commissioning nicht immer zuverlässig. Vorallem wenn viele Nodes gleichzeitig.}