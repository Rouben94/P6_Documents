\vspace*{4cm}
\part{Thread}\label{part:Thread}
Robin Bobst
\vspace*{\fill}
\clearpage

\section{Einleitung}\label{sec:EinleitungThread}
Thread ist ein auf IPv6 basiertes Netzwerkprotokoll, das speziell für Internet of Things (IoT) Anwendungen entwickelt wurde. Die einzelnen Teilnehmer im Netzwerk verbinden sich zu einem Mesh-Netzwerk. Wie in der Abbildung \ref{fig:ThreadProtokollLayer} ersichtlich verwendet Thread für eine effiziente Kommunikation mit IPv6-Paketen das Kommunikationsprotokoll 6LoWPAN (IPv6 over Low Power Personal Area Network). 6LoWPAN wendet ein Header-Kompressionsverfahren an, welches es ermöglicht die IPv6-Pakete über den Standard IEEE-802.15.4 zu übermitteln. Dank diesem Standard ist es machbar, die mit Thread entwickelten Geräte so energieeffizient zu gestalten, dass ein Batteriebetrieb realisierbar ist. In der Tabelle \ref{table:MerkmaleThread} sind die wichtigsten Merkmale von Thread aufgelistet. \\

\begin{figure}[H]
	\centering
	\includegraphics[width=0.5\textwidth]{threadlayerprotocols.png}
	\caption{Thread Protokoll Layer \cite{erickson_picture_2019}}
	\label{fig:ThreadProtokollLayer}
\end{figure}

% Please add the following required packages to your document preamble:
% \usepackage{booktabs}
\begin{table}[H]
	\centering
	\begin{adjustbox}{width=1\textwidth}
	\begin{tabular}{@{}lllll@{}}
		\cmidrule(r){1-2} \cmidrule(l){4-5}
		\multicolumn{2}{|c|}{\textbf{Netzwerk}}                                                              & \multicolumn{1}{l|}{} & \multicolumn{2}{c|}{\textbf{Applikation}}                                                                           \\ \cmidrule(r){1-2} \cmidrule(l){4-5} 
		\multicolumn{1}{|l|}{\textbf{Merkmal}}      & \multicolumn{1}{l|}{\textbf{Beschreibung}}             & \multicolumn{1}{l|}{} & \multicolumn{1}{l|}{\textbf{Merkmal}}                   & \multicolumn{1}{l|}{\textbf{Beschreibung}}                \\ \cmidrule(r){1-2} \cmidrule(l){4-5} 
		\multicolumn{1}{|l|}{IEEE 802.15.4}         & \multicolumn{1}{l|}{Protokoll}                         & \multicolumn{1}{l|}{} & \multicolumn{1}{l|}{IPv6}                               & \multicolumn{1}{l|}{IP-Kommunikation}                     \\ \cmidrule(r){1-2} \cmidrule(l){4-5} 
		\multicolumn{1}{|l|}{MAC Security}          & \multicolumn{1}{l|}{Verschlüsselte Übertragung}        & \multicolumn{1}{l|}{} & \multicolumn{1}{l|}{UDP}                                & \multicolumn{1}{l|}{UDP-Sockets}                          \\ \cmidrule(r){1-2} \cmidrule(l){4-5} 
		\multicolumn{1}{|l|}{6LoWPAN}               & \multicolumn{1}{l|}{Effiziente IPv6 Paket Übertragung} & \multicolumn{1}{l|}{} & \multicolumn{1}{l|}{CoAP}                               & \multicolumn{1}{l|}{Client und Server}                    \\ \cmidrule(r){1-2} \cmidrule(l){4-5} 
		\multicolumn{1}{|l|}{Mesh Routing}          & \multicolumn{1}{l|}{Many-to-many Kommunikation}        & \multicolumn{1}{l|}{} & \multicolumn{1}{l|}{DHCPv6}                             & \multicolumn{1}{l|}{Client und Server}                    \\ \cmidrule(r){1-2} \cmidrule(l){4-5} 
		&                                                        &                       &                                                         &                                                           \\ \cmidrule(r){1-2} \cmidrule(l){4-5} 
		\multicolumn{2}{|c|}{\textbf{Boarder Router}}                                                        & \multicolumn{1}{l|}{} & \multicolumn{2}{c|}{\textbf{Weitere Merkmale}}                                                                      \\ \cmidrule(r){1-2} \cmidrule(l){4-5} 
		\multicolumn{1}{|l|}{Web - UI}              & \multicolumn{1}{l|}{Für Netzwerk Menagement}           & \multicolumn{1}{l|}{} & \multicolumn{1}{l|}{Periodic parent search}             & \multicolumn{1}{l|}{Endgerät wechselt zu besserem Parent} \\ \cmidrule(r){1-2} \cmidrule(l){4-5} 
		\multicolumn{1}{|l|}{Externer Kommissioner} & \multicolumn{1}{l|}{Externes Gerät für Neuaufnahme}    & \multicolumn{1}{l|}{} & \multicolumn{1}{l|}{Jam Detection}                      & \multicolumn{1}{l|}{Signal Stau verhindern}               \\ \cmidrule(r){1-2} \cmidrule(l){4-5} 
		\multicolumn{1}{|l|}{NAT64}                 & \multicolumn{1}{l|}{Kommunikation mit IPv4}            & \multicolumn{1}{l|}{} & \multicolumn{1}{l|}{Child Supervision}                  & \multicolumn{1}{l|}{Endgerät überprüfen}                  \\ \cmidrule(r){1-2} \cmidrule(l){4-5} 
		\multicolumn{1}{|l|}{wpantund}              & \multicolumn{1}{l|}{Interface Treiber}                 & \multicolumn{1}{l|}{} & \multicolumn{1}{l|}{Inform previous parent on reattach} & \multicolumn{1}{l|}{Endgerät Menagement}                  \\ \cmidrule(r){1-2} \cmidrule(l){4-5} 
	\end{tabular}
	\end{adjustbox}
	\caption{Merkmale Thread}
	\label{table:MerkmaleThread}
\end{table}

	

	


In dieser Arbeit wird OpenThread verwendet. OpenThread ist eine open-sorce Implementation von Thread, die von Google umgesetzt und weiterentwickelt wird. 

