\vspace*{4cm}
\part{Mesh Benchmark - Resultate und Vergleich}\label{part:MeshBenchmarkResultateundVergleich}
\vspace*{\fill}
\clearpage

\section{Messresultate}\label{sec:Messresultate}
Bei der Durchführung der Mesh Benchmark Messungen wurde für jede Messung unter den entsprechenden Bedingungen ein Messprotokoll resp. eine Auswertung erstellt. Die acht unterschiedlichen Bedingungen sind in Tabelle \ref{tab:MessungenMeshBenchmark} nochmals zusammengefasst sind.
Diese detaillierten Auswertungen sind im Anhang \ref{app:MessprotokolleMeshBenchmark} dem Bericht angefügt.
Nachfolgend wird exemplarisch eine dieser Auswertungen erläutert, um aufzuzeigen, was diese darstellen und wie diese gelesen werden können.
Eine Interpretation der Messresultate im Rahmen eines Vergleichs erfolgt anschliessend im Abschnitt \ref{sec:VergleichMeshNetzwerke}.

\begin{table}[h]
\centering
\begin{tabular}{|c|c|c|c|c|c|} 
\hline
\textbf{\#}  & \textbf{Msg. Gen.}  & \textbf{Duration}  & \textbf{Msg. Cnt.}  & \textbf{Payload }  & \textbf{Disturbance}  \\ 
\hline
1 & Rand & 600s & 60 & Small & No \\ 
\hline
2 & Seq & 600s & 60 & Small & No \\ 
\hline
3 & Rand & 600s & 60 & Large & No \\ 
\hline
4 & Seq & 600s & 60 & Large & No \\ 
\hline
5 & Rand & 600s & 600 & Small & No \\ 
\hline
6 & Rand & 600s & 60 & Small & Yes \\ 
\hline
7 & Seq & 750s & 10 & Small & No \\ 
\hline
8 & Seq & 750s & 10 & Large & No \\
\hline
\end{tabular}
\caption{Messbedingungen Mesh Benchmark}
\label{tab:MessungenMeshBenchmark}
\end{table}


\subsection{Resultate}\label{subsec:Resultate}
Die Messresultate im Anhang \ref{app:MessprotokolleMeshBenchmark} wurden mit den Messindizes 1-8 gemäss Tabelle \ref{tab:MessungenMeshBenchmark} sowie der entsprechenden Messumgebung (z.B. Wohnung) versehen.
So können die Messungen eindeutig identifiziert werden.
Folgende Einschränkungen müssen dabei jedoch beachtet werden:
Gemäss den Erläuterungen im Abschnitt \ref{subsec:Messreihe} wurden die Messungen 6 - 8 nur im Labor-Messaufbau durchgeführt. Von diesen Messungen sind dementsprechend auch nur diese drei Auswertungen vorhanden.
Weiter musste bei der Durchführung der Messung 5 festgestellt werden, dass die Resultate unbrauchbar waren.
Aus diesem Grund musste die Auswertung dieser Messung gestrichen werden.
Im Abschnitt \ref{subsec:Validierung} wird nochmals darauf eingegangen.

Die Abbildungen \ref{fig:VerteilungderLatenzzeiten} bis \ref{fig:OngoingTransactions} zeigen die Messresultate der Messung 2 in der Messumgebung \textit{Wohnung}. Sie stehen exemplarisch für die Messergebnisse aller Messreihen.
In Abbildung \ref{fig:VerteilungderLatenzzeiten} ist die prozentuale Verteilung der Latenzzeiten pro Hop dargestellt.
In sämtlichen Grafiken werden die Ergebnisse von BT Mesh in blau, jene von Thread in grün und jene von Zigbee in rot dargestellt.
Dadurch ist ein direkter Vergleich der Protokolle möglich.
Abbildung \ref{fig:VerteilungderLatenzzeiten} zeigt folglich, wie viele Nachrichten das Ziel mit einer bestimmten Latenzzeit erreicht haben.
Nachrichten, die das Ziel nicht erreicht haben, also Pakete die verloren gegangen sind, werden dabei nicht berücksichtigt. Diese werden jedoch als Paketloss aufgezeichnet. Im gezeigten Beispiel haben rund 76 Prozent der Nachrichten, die im BT Mesh Test versendet wurden, das Ziel mit einer maximalen Latenzzeit von 10 Millisekunden erreicht.
Die weitere Verteilung geht bis auf Latenzzeiten von über 300 Millisekunden, wobei die Prozentzahl der Nachrichten in diesem Bereich sehr tief ist.
Die Werte für Thread und Zigbee zeigen hingegen eine deutlich kleinere Verteilung der Latenzzeiten.

\begin{figure}[h]
	\centering
	\includegraphics[width=\textwidth]{Latency_2_Wohnung.png}
	\caption{Messung 2 Wohnung: Verteilung der Latenzzeiten pro Hop}
	\label{fig:VerteilungderLatenzzeiten}
\end{figure}

Aus den in Abbildung \ref{fig:VerteilungderLatenzzeiten} aufgezeigten Latenzzeiten wurde der Durchschnitt über alle Messreihen gebildet und in Abbildung \ref{fig:DurchschnittlicheLatenzzeit} dargestellt.
Es handelt sich dabei wiederum um die Latenzzeit pro Hop. Im Falle von Zigbee ist dies erwähnenswert, da hier die Anzahl Hops nicht ausgelesen werden konnte (siehe Abschnitt \ref{subsec:ZigbeeSchwierigkeitenbeiderUmsetzung}) und die Resultate somit mit Vorsicht interpretiert werden müssen. Mehr dazu in der Validierung im Abschnitt \ref{subsec:Validierung}.

Der durchschnittliche Datendurchsatz, der in Abbildung \ref{fig:DurchschnittlicherDurchsatz} aufgezeigt wird, muss ebnfalls mit Vorsicht betrachtet werden. Denn auch hier werden die Werte pro Hop für die Berechnung verwendet.
Die präsentierten Werte werden aus der Paketgrösse (Small oder Large) gemäss den Definitionen in Abschnitt \ref{subsec:AllgemeineBenchmarkParameter} und der Latenzzeit des übertragenen Pakets (siehe Gleichung \ref{eq:BerechnungDurchsatz}), errechnet.
Dabei werden die Werte für den Durchsatz pro empfangenes Paket berechnet und davon schliesslich der Mittelwert gebildet.
Die oben erwähnten Ausreisser bei BT Mesh bewirken nun einen unerwartet hohen Durchsatz im Vergleich zu jenem bei Thread, das konstant tiefe Latenzzeiten aufweist.

\begin{equation}\label{eq:BerechnungDurchsatz}
TP =  \frac{S_{packet} \cdot 8}{t_{lat}}
\end{equation}

\begin{small}
\begin{center}
\begin{tabular}{ll}
$TP$ & Throughput (kBit/s)\\
$S_{packet}$ & Packetsize (Byte)\\
$t_{lat}$ & Latency time (ms)\\
\end{tabular}
\end{center}
\end{small}

\begin{figure}[!htbp]
\centering
\begin{minipage}[b]{0.49\textwidth}
		\centering
		\includegraphics[width=\textwidth]{Average_Latency_per_Hop.png}
		\caption{Messung 2 Wohnung: Durchschnittliche Latenzzeit pro Hop}
		\label{fig:DurchschnittlicheLatenzzeit}
\end{minipage}
\begin{minipage}[b]{0.49\textwidth}
		\centering
		\includegraphics[width=\textwidth]{Average_Throughput_per_Hop.png}
		\caption{Messung 2 Wohnung: Durchschnittlicher Durchsatz pro Hop}
		\label{fig:DurchschnittlicherDurchsatz}
\end{minipage}
\end{figure}

In Abbildung \ref{fig:DurchschnittlicherPaketverlust} wird der Paketverlust im Verhältnis zur gesamten Anzahl Nachrichten, die während dem Benchmark versendet wurden, in Prozenten aufgezeigt.
Die Paketverluste von einzelnen Client-Server Beziehungen werden nicht separat ausgewertet.
Wiederum im Beispiel von BT Mesh sind in dieser Messung 2.07 \% der Pakete nicht am Ziel angekommen.

\begin{figure}[!htbp]
\centering
\begin{minipage}[b]{0.49\textwidth}
		\centering
		\includegraphics[width=\textwidth]{Average_Packet_Loss.png}
		\caption{Messung 2 Wohnung: Durchschnittlicher Paketverlust}
		\label{fig:DurchschnittlicherPaketverlust}
\end{minipage}
\begin{minipage}[b]{0.49\textwidth}
		\centering
		\includegraphics[width=\textwidth]{Average_Radio_Energy_Consumption.png}
		\caption{Messung 2 Wohnung: Durchschnittlicher Energieverbrauch}
		\label{fig:DurchschnittlicherEnergieverbrauch}
\end{minipage}
\end{figure}

Mit dem Diagramm in Abbildung \ref{fig:DurchschnittlicherEnergieverbrauch} wird schliesslich noch der durchschnittliche Energieverbrauch der Protokolle dargestellt.
Dieser wird abgeleitet aus der \textit{Active Radio Time} (siehe Abschnitt \ref{subsubsec:Vergleichswerte}), welche direkt auf dem nRF52840 SoC mit der entsprechenden API ausgelesen werden kann.
Die \textit{Active Radio Time} wurde schliesslich mit dem Strombedarf des SoC's bei definierter Speisespannung von 3V verrechnet.
Gemäss den Angaben in Tabelle \ref{tab:EigenschaftennRF52840SoC} aus Abschnitt \ref{subsec:SystemonChip} beträgt der Strombedarf bei aktivem Funkmodul im Sendemodus 4.8mA.
Eine solche Berechnung erlaubt einen qualitativen Vergleich des Energiebedarfs unter den drei Protokollen, da die Umsetzung auf der gleichen Hardware erfolgte.
Die Werte in Abbildung \ref{fig:DurchschnittlicherEnergieverbrauch} sind jedoch keine quantitativen Verbrauchswerte und können deshalb nur im Kontext des Vergleichs verwendet werden.
Der Strombedarf sämtlicher Peripherie wurde nicht berücksichtigt, da dieser prinzipiell bei allen Protokollen identisch und in diesem Fall daher nicht interessant ist.

\begin{figure}[h]
	\centering
	\includegraphics[width=\textwidth]{Ongoing_Transactions_and_Latency.png}
	\caption{Messung 2 Wohnung: Ongoing Transactions und Verlauf der Latenzzeiten über die Messdauer.}
	\label{fig:OngoingTransactions}
\end{figure}

Die letzte Grafik gemäss Abbildung \ref{fig:OngoingTransactions} zeigt den Verlauf der \textit{Ongoing Transactions} sowie der Latenzzeiten über die Gesamtdauer einer Messung.
In diesem Fall beträgt die Dauer 600 Sekunden.
Die Grafik soll einen Eindruck darüber vermitteln, wie die Stacks damit umgehen, wenn vielen Nachrichten zur selben Zeit versendet werden.
Die \textit{Ongoing Transactions}, welche als durchgezogene Linien dargestellt sind, zeigen zu welchem Zeitpunkt wie viele Nachrichten im Übermittlungsprozess sind.
Da die Nachrichten in diesem Beispiel sequentiell versendet werden, gibt es nur sehr geringe Ausschläge, welche im unteren Bildrand der Abbildung \ref{fig:OngoingTransactions} zu sehen sind.
Die logarithmische Darstellung der Latenzzeiten als gestrichelte Linie bestätigt die Beobachtungen, die in Abbildung  \ref{fig:VerteilungderLatenzzeiten} bereits gemacht wurden.
Zigbee und Thread weisen einen ziemlich regelmässigen Verlauf der Latenzzeiten auf.
BT Mesh hingegen zeigt starke Ausreisser.

\subsection{Validierung}\label{subsec:Validierung}
Die durchgeführten Messungen haben aussagekräftige Resultate geliefert, welche jedoch stark vom gewählten Vorgehen abhängig sind.
Dieses Vorgehen wird nun kritisch überprüft und allfällige Mängel im Konzept sowie in der Umsetzung werden aufgezeigt.
Zudem werden systematische Messfehler deklariert.

\paragraph{Large Payload}
Die Messungen 3, 4 und 8 gemäss Tabelle \ref{tab:MessungenMeshBenchmark}, welche mit einer grossen Payload durchgeführt wurden, sind nur für Thread und Zigbee aussagekräftig. Der BT Mesh Stack liefert bei diesen Messungen keine brauchbaren Resultate.
Eine Recherche zu diesem Problem hat ergeben, dass durch die Fragmentierung einer 32 Byte Payload die sichere und schnelle Übertragung der Daten nicht mehr gewährleistet werden kann.

\todo[inline]{Raffi: Referenz auf Artikel wenn vorhanden. Raffi? Ansonsten Text anpassen.}

\paragraph{Zigbee Latency}
Wie bereits im Abschnitt \ref{subsec:ZigbeeSchwierigkeitenbeiderUmsetzung} erwähnt, konnte bei Zigbee die Anzahl Hops, die ein Paket passiert hat nicht ausgewertet werden.
Der Forumsbeitrag \href{https://devzone.nordicsemi.com/f/nordic-q-a/63815/zigbee---read-number-of-hops-radius}{\textit{Zigbee - Read number of hops (radius)\footnote{\url{https://devzone.nordicsemi.com/f/nordic-q-a/63815/zigbee---read-number-of-hops-radius}}}} bestätigt, dass der entsprechende Wert mit der verwendeten SDK nicht ausgelesen werden kann.
Die Folge davon ist, dass die Latenzzeit nur als Total über die gesamte Strecke erfasst werden kann.
In der Auswertung verschafft dies fälschlicherweise BT Mesh und Thread einen Vorteil gegenüber Zigbee.
Die Auswertung der totalen Latenzzeit bei allen drei Protokollen könnte das Problem lösen.
Dies würde jedoch dem Messkonzept widersprechen und wurde deshalb nicht für alle Messungen umgesetzt.
Die Abbildungen \ref{fig:DurchschnittlicheLatenzzeitValidierung} und \ref{fig:DurchschnittlicheLatenzzeitohneHopsValidierung} zeigen den Unterschied am Beispiel der oben analysierten Messung im Abschnitt \ref{subsec:Resultate}.
Während in Abbildung \ref{fig:DurchschnittlicheLatenzzeitValidierung} die Latenzzeit pro Hop dargestellt ist, zeigt die Abbildung \ref{fig:DurchschnittlicheLatenzzeitohneHopsValidierung} die totale Latenzzeit.
Der Unterschied ist vorallem bei Thread deutlich erkennbar.
Bereits in Abbildung \ref{fig:VerteilungderLatenzzeiten} ist diese Problematik zu erahnen.
Denn die Statistik der Latenzzeiten von Zigbee zeigt zwei Hauptsäulen bei 40ms und 70ms. Dies deutet darauf hin, dass die Pakete mit 70ms Latenzzeit einen Hop passiert haben und jene bei 40ms nicht.


\begin{figure}[!htbp]
\centering
\begin{minipage}[b]{0.49\textwidth}
		\centering
		\includegraphics[width=\textwidth]{Average_Latency_per_Hop.png}
		\caption{Durchschnittliche Latenzzeit pro Hop}
		\label{fig:DurchschnittlicheLatenzzeitValidierung}
\end{minipage}
\begin{minipage}[b]{0.49\textwidth}
		\centering
		\includegraphics[width=\textwidth]{Average_Latency_without_Hops.png}
		\caption{Durchschnittliche Latenzzeit ohne Berücksichtigung der Hops.}	\label{fig:DurchschnittlicheLatenzzeitohneHopsValidierung}
\end{minipage}
\end{figure}

\paragraph{Nachrichten Dichte}
Bei der Definition der Messreihen \ref{subsec:Messreihe} wurden zu Beginn nur die Messungen 1 bis 6 spezifiziert.
Die Messreihen 7 und 8 kamen erst nachträglich hinzu, als festgestellt werden musste, dass die Dichte der Nachrichten für den BT Mesh Stack zu hoch gewählt wurde.
Der BT Mesh Stack war überfordert beim Empfang von zufällig generierten Nachrichten, wie im Abschnitt \ref{subsec:TrafficGeneration} beschrieben ist.
Thread und Zigbee zeigten indes keine Mühe mit der Dichte von 60 Nachrichten in 600 Sekunden pro Node.
Die Resultate der Messungen 7 und 8 haben schliesslich gezeigt, dass die Reduktion der Nachrichtendichte einen positiven Einfluss auf die Performance von BT Mesh hat.

Die Messung Nummer 5 wurde aufgrund der selben Problematik aus den Auswertungen gestrichen. 
Hier wäre die Dichte der Nachrichten um das zehnfache grösser gewesen als bei der vergleichbaren Messung mit dem Index 1.
Eine Auswertung der Messdaten wäre daher sinnlos gewesen.

\paragraph{Group addressing mode}
Der \textit{Group addressing mode} wurde im Abschnitt \ref{subsec:AllgemeineBenchmarkParameter} für die drei Protokolle unterschiedlich definiert.
Erste Tests vor den eigentlichen Benchmarks haben gezeigt, dass eine Multicast Adressierung bei Zigbee unbrauchbar ist (siehe Abschnitt \ref{subsec:ZigbeeSchwierigkeitenbeiderUmsetzung}).
Deshalb hat man sich entschieden, bei Zigbee eine Unicast Adressierung umzusetzen.
Während den Benchmarks musste schliesslich festgestellt werden, dass Zigbee durch diese Änderung einen wesentlichen Vorteil erlangt hat.
Besonders auf die Paketverlustrate hat sich die Unicast Adressierung positiv ausgewirkt, denn Unicast Nachrichten verursachen deutlich weniger Verkehr innerhalb des Netzwerkes als Broadcast Nachrichten.

Auch in der Messung Nr. 5 hätte sich die Unicast Adressierung für Zigbee positiv ausgewirkt, da die Netzbelastung deutlich geringer gewesen wäre.
 
\paragraph{Durchschnittswerte in den Resultaten}
In den Resultaten \ref{subsec:Resultate} wurden sämtliche Durchschnittswerte als Mittelwerte einschliesslich allfälliger Ausreisser aus den Messwerten gebildet.
Dadurch wurden gewisse Resultate deutlich verfälscht.
In einer verbesserten Auswertung müsste die Ursache für die einzelnen Ausreisser genauer geklärt werden und diese allenfalls gestrichen werden. Zudem müsste für eine bessere Auswertung der Median bestimmt werden, um allfällige Ausreisser zu kompensieren.
Dies wurde leider zu spät entdeckt und konnte daher nicht umgesetzt werden.


\subsection{Verifizierung}\label{subsec:Verifizierung}
Eine Verifizierung der Messresultate konnte nur anhand des Referenzberichts \textit{AN1142: Mesh Network Performance
Comparison\footnote{\url{https://www.silabs.com/documents/public/application-notes/an1142\%2Dmesh\%2Dnetwork\%2Dperformance\%2Dcomparison.pdf} \cite{silicon_laboratories_inc_an1142_2020}}} von \textit{Silicon Labs} gemacht werden.
Dieser ist auf der Webseite von \textit{Silicon Labs} einsehbar.

Der Vergleich der Messergebnisse hat gezeigt, dass die Grössenordnung der Resultate mit jenen aus dem Bericht von Silicon Labs übereinstimmt.
Selbst die Ausreisser in der Latenzzeit bei BT Mesh liegen im selben Rahmen.
Zudem kann die klare Abhängigkeit der Resultate von der Grösse der Payload bestätigt werden.
Einige Unterschiede sind jedoch in der Verteilung der Latenzzeiten erkennbar.
Im Referenzbericht ist diese in einem Bereich zwischen 20ms und 60ms ziemlich regelmässig, während in den Ergebnissen dieser Thesis die Verteilung unregelmässiger daherkommt.






