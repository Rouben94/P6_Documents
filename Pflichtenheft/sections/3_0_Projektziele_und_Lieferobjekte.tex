\clearpage
\section{Projektziele und Lieferobjekte}\label{sec:ProjektzieleundLieferobjekte}

Nachfolgend sind alle Projektziele aufgelistet. Die Ziele wurden in drei verschiedene Themen aufgeteilt.

\subsection{Punkt zu Punkt Testinfrastruktur}\label{subsec:PunktzuPunktTestinfrastruktur}

\begin{table}[H]
	\begin{tabular}{ | C{0.7cm} | p{2.8cm} | p{8.2cm} | C{2.5cm} |}
		\hline
		\multicolumn{4}{|l|}{\textbf{Projektziele}}\\ \hline
		\textbf{Nr.}& \textbf{Ziel}& \textbf{Beschrieb}& \textbf{\begin{tabular}[b]{@{}c@{}}Pflicht- oder\\Wunschziel\end{tabular}}\\ \hline
		
		P1 & Kommunikation mit Leitrechner & Das Master Funkmodul kommuniziert via USB oder UART mit dem Leitrechner.  & Pflichtziel\\ \hline
		
		P2 & Absenden von MAC-Frames & Das Master-Funkmodul sendet gemäss den Vorgaben des Leitrechners regelmässig MAC-Frames an einen oder mehrere Slave-Funkknoten. & Pflichtziel\\ \hline
		
		P3 & Rückbestätigung der MAC-Frames & Der oder die batteriebetriebenen Slave-Funkknoten, bestätigen die MAC-Frames zurück. & Pflichtziel\\ \hline
		
		P4 & Konfiguration der Anzahl Slave-Knoten & Die Anzahl der Slave-Funkknoten ist über eine Steuer- und Auswertesoftware konfigurierbar & Pflichtziel\\ \hline
		
		P5 & Konfiguration der ID der Slave-Knoten & Die ID der Slave-Funkknoten ist über eine Steuer- und Auswertesoftware konfigurierbar & Pflichtziel\\ \hline
		
		P6 &Konfiguration der Kanäle & Die BLE resp. 802.15.4 Kanäle können über eine Steuer- und Auswertesoftware ausgewählt werden. & Pflichtziel\\ \hline
		
		P7 & Einstellbare Framelänge & Die Framelänge der MAC-Frames ist über eine Steuer- und Auswertesoftware konfigurierbar & Pflichtziel\\ \hline
		
		P8 & Einstellbare Frame- und Kanalwechselrate& Die Frame- und Kanalwechselrate der Funkknoten ist über eine Steuer- und Auswertesoftware konfigurierbar & Pflichtziel\\ \hline
		
		P9 & Einstellbare Sendeleistung & Die Sendeleistung der Funkknoten ist über eine Steuer- und Auswertesoftware konfigurierbar & Pflichtziel\\ \hline
		
		P10 & Anpassung der Modulationsart& In BLE soll die Modulationsart über eine Steuer- und Auswertesoftware konfigurierbar sein, um die Datenrate von 125kbps auf 2Mbps und die Long Range Funktion einzustellen & Pflichtziel\\ \hline
		
		P11 & Ein- und Ausschalten der Collision Avoidance (CSMA/CA)& Beim 802.15.4 Protokoll soll die Collision Avoidance über eine Steuer- und Auswertesoftware ein- und ausgeschaltete werden können& Pflichtziel\\ \hline
		
		P12 & Erfassen der Verbindungsqualität & Sowohl master- wie auch slaveseitige Erfassung der Verbindungsqualität (RSSI, Package Loss, Collisions, Noise Level, …). Die Slaves senden hierzu die erfassen Werte im Rückantwortframe dem Master zurück. & Pflichtziel\\ \hline
		
	\end{tabular}\\
	\caption{Projektziele der Punkt zu Punkt Testinfrastruktur}
	\label{tab:ProjektzielederPunktzuPunktTestinfrastruktur}
\end{table}


\subsection{Test Mesh Netzwerke}\label{subsec:TestMeshNetzwerke}
\todo[inline]{Robin}
\begin{table}[H]
	\begin{tabular}{ | C{0.7cm} | p{2.8cm} | p{8.2cm} | C{2.5cm} |}
		\hline
		\multicolumn{4}{|l|}{\textbf{Projektziele}}\\ \hline
		\textbf{Nr.}& \textbf{Ziel}& \textbf{Beschrieb}& \textbf{\begin{tabular}[b]{@{}c@{}}Pflicht- oder\\Wunschziel\end{tabular}}\\ \hline
		
		P1 & Ansteuerung Funkmodul & Der Leitrechner steuert via USB oder UART ein Master-Funkmodul an.  & Pflichtziel\\ \hline
		
		P1 &Konfiguration Mesh Netzwerk &Das Master-Funkmodul konfiguriert über das Mesh-Netzwerk die verschiedenen Nodes anhand der Daten vom Leitrechner. & Pflichtziel\\ \hline
		
		P1 &Konfiguration Mesh Netzwerk & Der Leitrechner steuert via USB oder UART ein Master Funkmodul an.  & Pflichtziel\\ \hline
		
	\end{tabular}\\
	\caption{Projektziele der Test Mesh Netzwerke}
	\label{tab:ProjektzielederTestMeshNetzwerke}
\end{table}

\subsection{Steuer und Auswertesoftware}\label{subsec:SteuerundAuswertesoftware}
\todo[inline]{Robin}
\begin{table}[H]
	\begin{tabular}{ | C{0.7cm} | p{2.8cm} | p{8.2cm} | C{2.5cm} |}
		\hline
		\multicolumn{4}{|l|}{\textbf{Projektziele}}\\ \hline
		\textbf{Nr.}& \textbf{Ziel}& \textbf{Beschrieb}& \textbf{\begin{tabular}[b]{@{}c@{}}Pflicht- oder\\Wunschziel\end{tabular}}\\ \hline
		
		P1 & Ansteuerung Funkmodul & Der Leitrechner steuert via USB oder UART ein Master Funkmodul an.  & Pflichtziel\\ \hline
		
		P2 & Visualisierung Parameter& Die Parameter der Ziele von Kapitel \ref{subsec:TestMeshNetzwerke} und \ref{subsec:PunktzuPunktTestinfrastruktur} sollen vom Leitrechner visualisiert und eingestellt werden können. & Pflichtziel\\ \hline
		
		P3 & Konfiguration Mesh-Netzwerk& Der Leitrechner verwaltet und konfiguriert über einen Master-Funkmodul das Mesh-Netzwerk. & Pflichtziel\\ \hline
		
		P4 & Auswertung der gemessenen Daten & Der Leitrechner soll die erhaltenen Daten auswerten und anzeigen.  & Pflichtziel\\ \hline
		
		P2 & Gliederfüßler & Das Glied muss unbedingt eingeführt werden & Pflichtziel\\ \hline
		
		
		
	\end{tabular}\\
	\caption{Projektziele der Steuer und Auswertesoftware}
	\label{tab:ProjektzielederSteuerundAuswertesoftware}
\end{table}





\clearpage
\subsection{Lieferobjekte}\label{subsec:Lieferobjekte}
\todo[inline]{Robin}

Zusätzlich zu den Projektzielen, folgen in diesem Kapitel die Lieferobjekte  mit dem jeweiligen Datum. In der Tabelle \ref{tbl:Lieferobjekte} sind diese  aufgelistet.  


\begin{table}[H]
     \centering
\begin{tabular}{|c|c|l|}\hline
   \textbf{Nr.} & \textbf{Datum} & \textbf{Lieferobjekt} \\ \hline
   
   1 & 02.03.2020 & Abgabe Pflichtenheft, 1. Version\\ \hline
   2 & 08.03.2020 & Abgabe Pflichtenheft, definitive Version\\ \hline
   3 & 14.08.2020 & Abgabe Fachbericht \\ \hline
   4 & 14.08.2020 & Abgabe Paper \\ \hline
   5 & 14.08.2020 & Abgabe Testaufbau \\ \hline
   6 & 01.09.2020 & Projektpräsentation \\ \hline
   
   
   
 \end{tabular}
     \caption{Lieferobjekte}
     \label{tbl:Lieferobjekte}
\end{table}