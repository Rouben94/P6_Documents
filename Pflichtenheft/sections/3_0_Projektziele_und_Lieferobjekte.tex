\clearpage
\section{Projektziele und Lieferobjekte}\label{sec:ProjektzieleundLieferobjekte}

Nachfolgend sind alle Projektziele aufgelistet. Die Ziele wurden in vier Teile unterteilt. Die ersten drei Teile sind entsprechend den Hauptzielen die in Kapitel \ref{subsec:ZielderArbeit} definiert. Die Tabelle \ref{tab:ZusatzzieledesGesamtprojektes} zeigt zusätzliche Ziele die als Wunschziele betrachtet werden und somit nicht im Fokus der Arbeit stehen.

\subsection{Punkt zu Punkt Testinfrastruktur}\label{subsec:PunktzuPunktTestinfrastruktur}

\begin{table}[H]
    \centering
	\begin{tabular}{ | C{0.7cm} | p{3.7cm} | p{10.2cm} |}
		\hline
		\multicolumn{3}{|l|}{\textbf{Projektziele}}\\ \hline
		\textbf{Nr.}& \textbf{Ziel}& \textbf{Beschrieb}\\ \hline
		
		P1 & Kommunikation mit BMS & Das BMN kommuniziert via USB oder UART mit dem BMS.\\ \hline
		
		P2 & Absenden von MAC-Frames & Das BMN sendet gemäss den Vorgaben des BMS regelmässig MAC-Frames an einen oder mehrere BSN.\\ \hline
		
		P3 & Rückbestätigung der MAC-Frames & Der oder die batteriebetriebenen BSN, bestätigen die MAC-Frames zurück.\\ \hline
		
		P4 & Konfiguration der Anzahl BSN & Die Anzahl der BSN ist über eine Steuer- und Auswertesoftware konfigurierbar.\\ \hline
		
		P5 & Konfiguration der ID der BSN & Die ID der BSN ist über eine Steuer- und Auswertesoftware konfigurierbar.\\ \hline
		
		P6 &Konfiguration der Kanäle & Die BLE resp. 802.15.4 Kanäle können über eine Steuer- und Auswertesoftware ausgewählt werden.\\ \hline
		
		P7 & Einstellbare Framelänge & Die Framelänge der MAC-Frames ist über eine Steuer- und Auswertesoftware konfigurierbar.\\ \hline
		
		P8 & Einstellbare Frame- und Kanalwechselrate& Die Frame- und Kanalwechselrate der BSN ist über eine Steuer- und Auswertesoftware konfigurierbar.\todo[inline]{nicht nur beim BSN}\\ \hline
		
		P9 & Einstellbare Sendeleistung & Die Sendeleistung der BSN ist über eine Steuer- und Auswertesoftware konfigurierbar.\\ \hline
		
		P10 & Anpassung der Modulationsart& In BLE soll die Modulationsart über eine Steuer- und Auswertesoftware konfigurierbar sein, um die Datenrate von 125kbps auf 2Mbps und die Long Range Funktion einzustellen.\\ \hline
		
		P11 & Ein- und Ausschalten der Collision Avoidance (CSMA/CA)& Beim 802.15.4 Protokoll soll die Collision Avoidance über eine Steuer- und Auswertesoftware ein- und ausgeschaltete werden können.\\ \hline
		
		P12 & Erfassen der Verbindungsqualität & Sowohl master- wie auch slaveseitige Erfassung der Verbindungsqualität (RSSI, Package Loss, Collisions, Noise Level, …). Die BSN senden hierzu die erfassten Werte im Rückantwortframe dem BMN zurück.\\ \hline

		
	\end{tabular}\\
	\caption{Projektziele der Punkt zu Punkt Testinfrastruktur}
	\label{tab:ProjektzielederPunktzuPunktTestinfrastruktur}
\end{table}


\subsection{Test Mesh Netzwerke}\label{subsec:TestMeshNetzwerke}
\begin{table}[H]
\centering
	\begin{tabular}{| C{0.7cm} | p{3.7cm} | p{10.2cm} |}
		\hline
		\multicolumn{3}{|l|}{\textbf{Projektziele}}\\ \hline
		\textbf{Nr.}& \textbf{Ziel}& \textbf{Beschrieb}\\ \hline
		
		P1 & Kommunikation mit BMS & Das BMN kommuniziert via USB oder UART mit dem BMS.\\ \hline
		
		P2 &Konfiguration BSN &Die BSN lassen sich frei zu einem Routing-Knoten, End-Knoten oder einem Low-Power Knoten konfigurieren.\\ \hline
		
		P3 & Mesh-Netzwerk & Alle drei Technologien Bluetooth, Thread und Zigbee müssen als Mesh-Netzwerk mit mindestens 10 BSN aufgebaut werden.\\ \hline
		
		P4 & Simulation Sensorwerte & Die BSN sollen in einem vom BMS vorgegebenen parametrisierbaren Intervall Sensorwerte Simulieren\\ \hline
		
		P5 & Sensordaten & Als Sensordaten sollen die Netzzustandsdaten übermittelt werden: Paketnummer, Anzahl Retries, Paketverluste, RSSI, Strombedarf und aktive CPU- und Radio-Zeiten.\\ \hline
		
		P6 & Störimmunität & Um die Störimmunität der Netzwerke zu ermitteln sollen gezielt Fremdstörungen mit eingebracht werden, mit definierbarer Tastung und Störframelänge. Hierfür sollen die Punkt zu Punkt Testinfrastrukturen auf MAC-Ebene eingesetzt werden.\\ \hline
		
		P7 & Test und Validierung & Umfassende Gegenüberstellung und Validierung aller drei Netzwerke in einem FHNW Gebäude, insbesondere Durchsatz, Antwortzeit, Zuverlässigkeit, Einfachheit der Konfiguration (inkl. Routing), Einfachheit der Ermittlung geeigneter Router-Standorte, Sicherheit und Energieverbrauch.\\ \hline
		
	\end{tabular}\\
	\caption{Projektziele der Test Mesh Netzwerke}
	\label{tab:ProjektzielederTestMeshNetzwerke}
\end{table}


\subsection{Steuer- und Auswertesoftware}\label{subsec:SteuerundAuswertesoftware}
\begin{table}[H]
\centering
	\begin{tabular}{| C{0.7cm} | p{3.7cm} | p{10.2cm} |}
		\hline
		\multicolumn{3}{|l|}{\textbf{Projektziele}}\\ \hline
		\textbf{Nr.}& \textbf{Ziel}& \textbf{Beschrieb}\\ \hline
		
		P1 & Ansteuerung Funkmodul & Das BMS steuert via USB oder UART ein BMN an.\\ \hline
		
		P2 & Visualisierung Parameter& Die Parameter der Ziele von Kapitel \ref{subsec:TestMeshNetzwerke} und \ref{subsec:PunktzuPunktTestinfrastruktur} sollen vom BMS visualisiert und eingestellt werden können.\\ \hline
		
		P3 & Konfiguration Mesh-Netzwerk& Das BMS verwaltet und konfiguriert über einen BMN das Mesh-Netzwerk.\\ \hline
		
		P4 & Einheitliche Kommunikation von BMS & Das Protokoll und Interface zum BMS soll für alle drei Mesh-Netzwerke einheitlich sein.\\ \hline
		
	\end{tabular}\\
	\caption{Projektziele der Steuer- und Auswertesoftware}
	\label{tab:ProjektzielederSteuerundAuswertesoftware}
\end{table}

\subsection{Zusatzziele}\label{subsec:Zusatzziele}
\begin{table}[H]
\centering
	\begin{tabular}{| C{0.7cm} | p{3.7cm} | p{10.2cm} |}
		\hline
		\multicolumn{3}{|l|}{\textbf{Projektziele}}\\ \hline
		\textbf{Nr.}& \textbf{Ziel}& \textbf{Beschrieb}\\ \hline
		
		
		W1 & Testmodul BMN/BSN & Entwickeln eines Testmoduls mit unabhängiger Stromversorgung und dezidierter Strommessung um den Stromverbrauch aufzuzeichnen.\\ \hline
		
		W2 & Vergleich SOC & Vergleichen zwischen nRF52840, nRF5340 und weiterer kompatibler SOCs.\todo[inline]{Evtl. nicht ein Bootloader sondern zusammenführen der 3 Mesh Protokolle und Softwaremässig umschaltbar machen.}\\ \hline
		
		W3 & Drahtlos Konfiguration & Konfiguration der BSN / BMN Firmware drahtlos über Bootloader.\\ \hline
		
	\end{tabular}\\
	\caption{Zusatzziele des Gesamtprojektes}
	\label{tab:ZusatzzieledesGesamtprojektes}
\end{table}

\todo[inline]{Portierbares Framework für die Ausmessung eines Netzes. Mit Darstellung der Signal Qualität auf den verschiedenen Kanälen.}
\todo[inline]{GUI für Mesh Test als Zusatzziel}


\subsection{Lieferobjekte}\label{subsec:Lieferobjekte}
Zusätzlich zu den Projektzielen, folgen in diesem Kapitel die Lieferobjekte  mit dem jeweiligen Fälligkeitsdatum. In der Tabelle \ref{tbl:Lieferobjekte} sind diese  aufgelistet.  


\begin{table}[H]
     \centering
\begin{tabular}{|c|c|l|}\hline
   \textbf{Nr.} & \textbf{Datum} & \textbf{Lieferobjekt} \\ \hline
   
   1 & 02.03.2020 & Abgabe Pflichtenheft, 1. Version\\ \hline
   2 & 08.03.2020 & Abgabe Pflichtenheft, definitive Version\\ \hline
   3 & 14.08.2020 & Abgabe Fachbericht \\ \hline
   4 & 14.08.2020 & Abgabe Paper \\ \hline
   5 & 14.08.2020 & Abgabe Testaufbau \\ \hline
   6 & 14.08.2020 & Abgabe Factsheet \\ \hline
   7 & 14.08.2020 & Abgabe Poster \\ \hline
   8 & 01.09.2020 & Projektpräsentation \\ \hline
   
 \end{tabular}
     \caption{Lieferobjekte}
     \label{tbl:Lieferobjekte}
\end{table}