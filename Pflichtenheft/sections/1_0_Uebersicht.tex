	\clearpage
\section{Übersicht}\label{sec:Uebersicht}

Das vorliegende Dokument stellt das Pflichtenheft der Bachelorthesen von Raffael Anklin, Robin Bobst und Cyrill Horath an der Fachhochschule Nordwestschweiz Brugg-Windisch im Studiengang Elektro- und Informationstechnik dar. 
Im kommenden, ersten Kapitel soll eine Übersicht über die Ausgangslage sowie das Ziel dieser Arbeit gegeben werden und somit die Rahmenbedingungen abgesteckt werden. Weiter werden die Lösungskonzepte \ref{sec:Loesungskonzept} sowie die Projektziele und Lieferobjekte \ref{sec:ProjektzieleundLieferobjekte} definiert. Abschliessend soll auch noch das Projektmanagement \ref{sec:Projektmanagement} thematisiert werden. 

\subsection{Ausgangslage}\label{subsec:Ausgangslage}

Unter den standardisierten Low Power Mesh Netzwerk Protokollen im
freien GHz ISM-Band konkurrenzieren sich derzeit vorrangig Bluetooth Mesh, Zigbee sowie Thread.
Bezüglich MAC und Physical Layer basieren Zigbee und Thread auf IEEE 802.15.4 wogegen Bluetooth Mesh auf Bluetooth Low Energy (BLE)
basiert.
Jedes dieser Netzwerkprotokolle hat gewisse Vorzüge: Bluetooth Mesh, dass BLE mittlerweile von jedem Smartphone und Notebook unterstützt wird, Thread aufgrund seiner IPv6 Basis und damit einfachem Übergang ins Internet sowie Zigbee aufgrund seiner etablierten Verbreitung im Smart-Lampenbereich durch Philips, IKEA und Osram.
Hauptproblem aller drei Mesh Netzwerkprotokolle ist nebst physikalisch und distanzbedingter Absorption und Reflexion die Störbeeinflussung durch WLAN (WiFi) und andere Netzwerke im GHz Frequenzbereich.

Im Rahmen des P5 mit dem Namen \textit{Bluetooth-Mesh Plattform für IoT Anwendungen}, wurde das Bluetooth-Mesh Protokoll bereits vertieft betrachtet und dessen Vor- und Nachteile aufgezeigt. Basierend auf diesen Erkenntnissen und Erfahrungen und der oben beschriebenen Thematik soll das Bluetooth-Mesh Protokoll mit den Alternativen Thread sowie Zigbee verglichen werden.

\subsection{Ziel der Arbeit}\label{subsec:ZielderArbeit}

In der vorliegenden Arbeit soll zuerst ein praxistaugliches, einheitliches Testframework für alle drei Mesh Netzwerke erstellt werden, wonach die Tauglichkeit aller drei Mesh Netzwerke unter realitätsnahen Bedingungen ermittelt und verglichen werden soll.
Zwecks besserer Vergleichbarkeit sollen alle drei Testnetze das gleiche Radio-Interface als Grundlage verwenden. Aufgrund der guten
Unterstützung aller drei Mesh Protokolle als auch dem im vergangenen P5 gesammelten Wissens, sollen hierfür die nRF52840 SoCs der Firma Nordic eingesetzt werden. Die zu erstellende Testinfrastruktur soll aus den drei folgenden Teilen bestehen:

\begin{itemize}
 	\item Punkt-Punkt Testinfrastrukturen auf MAC-Ebene
 	\item Test Mesh Netzwerke für BT Mesh, Zigbee und Thread
 	\item Steuer- und Auswertesoftware
\end{itemize}

Die genauen Anforderungen an die Testumgebung sind einerseits in der Aufgabenstellung im Anhang \ref{app:Aufgabenstellung} aufgeführt und andererseits werden sie anhand der Projektziele \ref{sec:ProjektzieleundLieferobjekte} definiert.








