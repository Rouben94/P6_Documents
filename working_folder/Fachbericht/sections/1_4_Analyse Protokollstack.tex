\clearpage
\section{Analyse Bluetooth Mesh Stack}\label{sec:AnalyseBluetoothMeshStack}
Bluetooth-Mesh ist ein relativ neuer Stack mit wenig Anwendungsbeispielen. Er wurde entwickelt um mit der Konkurrenz wie Z-Wave oder Zigbee mithalten zu können. Im folgenden werden die Grenzen des Stacks aufgezeigt.


\subsubsection{Grenzen des Stacks}\label{subsec:BLEMeshProtokollStack}
Die Topologie eines Netzwerks beeinträchtigt stark seine Performance. Ist die Node-Dichte sehr hoch, steigt die Wahrscheinlichkeit, dass es zu Kollisionen der einzelnen Nachrichten kommt. Der BLE-MAC-Layer verfügt im Gegensatz zum IEEE 802.15.4 Layer über keine Collision Avoidance. Somit ist es im Interesse aller Teilnehmer, wenn die Radio-Sende-Zeiten so kurz wie möglich gehalten werden. Deshalb schreibt der Stack lediglich 29 Bytes als maximale Network-PDU vor. Durch die Geschwindigkeit des MAC-Layers von 1-2Mbit/s sinkt die Wahrscheinlichkeit von Kollisionen der Pakete weiter. Das versenden von längeren Nachrichten, welche stark segmentiert werden müssen, erhöhen die Gesamtbelastung des Netzwerks. \cite{bluetooth_sig_mesh_netzwerk_spezifikationen_2020}

Der Mesh-Stack schreibt eine maximale Grösse von 32767 Teilnehmern vor. Dies soll die theoretische sowie praktische Grenze darstellen. Ein Gebäude der Firma Silvair wurde mit über 1000 Bluetooth-Mesh-Nodes ausgestattet. Laut dem Betreiber laufe das System einwandfrei. Dabei handelt es sich um ein Office-Gebäude mit vielen anderen Bluetooth-Geräten wie Smartphones, Mäuse und Headsets sowie WLAN-Access-Points.  \cite{woolley_how_bluetooth_mesh_puts_the_large__2018}

Ein Bluetooth-Mesh Paket wird immer auf drei Kanälen (37,38,39) gesendet (BLE-GAP Advertising). Dadurch erhöht sich die Störfestigkeit. Jedoch verringert sich der Durchsatz mit dem Faktor drei, da ein Paket dreimal solange braucht um gesendet zu werden. 

\subsubsection{Bluetooth Mesh Implementation in Zephyr}\label{subsec:BLEMeshProtokollStackZephyr}
Die Implementation des Bluetooth-Mesh-Stacks in Zephyr weist einige Mängel auf. Es liegen Probleme mit dem HCI-Treiber (Host Controller Interface), welcher das Radio-Interface ansteuert, vor. Diese äussern sich dadurch, dass nach jedem Advertising (Senden) eines Mesh-Pakets  der aktive Thread für 10ms pausiert wird. Dadurch erhält die Abarbeitung des Stacks eine gravierende Verzögerung, bis wieder in den Scanner (Empfangs) Betrieb gewechselt werden kann. Vermutlich führt diese Lücke ebenfalls zum verkürzen der Aktive-Radio-Time, welche gemessen wird um den Energieverbrauch abzuschätzen. Sie gibt an wie lange das Radio-Interface aktiv genutzt wurde. 

Zudem wurde einen Problematik beim Segmentieren von Nachrichten in der Bluetooth-Mesh Implementation in Zephyr festgestellt. Beim Empfangen von segmentierten Nachrichten, werden diese zum Teil mehrmals empfangen. Ob dies Ebenfalls einen Einfluss auf das Messresultat hatte ist unklar. Ein Online\footnotemark\ Beitrag verweist ebenfalls auf diese Problematik. 

\footnotetext{\url{https://github.com/zephyrproject-rtos/zephyr/issues/16886}}


