\clearpage
\section{Analyse Thread Stack}\label{sec:AnalyseThreadStack}
In diesem Kapitel wird die persönliche Erfahrung mit dem Mesh-Stack zu Tage gelegt, die während des ganzen Entwicklungsprozesses entstanden ist.

\subsection{Vor- und Nachteile Openthread Stack}\label{susec:ThreadVorNachteile}
Nachfolgend werden die wichtigsten Vor- und Nachteile aufgezeigt, die sich als persönliche Erfahrungen beschreiben lassen.
\paragraph{Vorteile}
Der Openthread Stack von Google Nest ist auf der Webseite \href{https://openthread.io/}{Openthread\footnote{\url{https://openthread.io/}}} sehr gut beschrieben. Es gibt eine kurze Einleitung wie der Stack funktioniert, ohne dabei zu grob ins Detail zu gehen. Weiter sind fantastische Tutorials auf der Webseite verfügbar, mit denen man beginnt ein Thread Netzwerk zu simulieren. Danach wird man Schritt für Schritt angeleitet, wie man  mit einem SoC ein reales Netzwerk aufbaut. Zum Schluss, wenn man das minimale Netzwerk erfolgreich aufgebaut hat, wird man auf die API verwiesen. Diese ist auch sehr ausführlich dokumentiert und man kann alle Funktionen, die benötigt werden, mit einer Suchfunktion herausfinden.

\paragraph{Nachteile}
Ein grosser Nachteil des Projekts von Google Nest ist die Beschreibung der API. Die Dokumentation ist zwar lückenlos, jedoch wäre es sehr praktisch einige Beispiele der wichtigsten Funktionen zu haben. Meistens ist es schwierig, die API Funktionen zu implementieren, da mehrere Parameter eingestellt werden müssen. Dies könnte Google Nest in einem kurzen Beispiel gut ersichtlich machen. Ein weiterer Nachteil ist das Auslesen der Message ID. Dies ist nicht immer zuverlässig möglich. Dadurch musste mit der Payload der Benchmark Message eine Message ID mitgegeben werden. Aus diesem Grund wurde die Messung mit Acknowledgement nicht umgesetzt, da die Message ID nicht mit der Acknowledge Nachricht mitgegeben werden konnte. Daraus resultiert, dass die Nachrichten nicht identifiziert werden können.

\subsection{FreeRTOS Projekt}\label{susec:FreeRTOS Projek}
Auf dem \href{https://github.com/Rouben94/P6_Software}{Github-Repository\footnote{\url{https://github.com/Rouben94/P6_Software}\cite{anklin_bobst_horath_rouben94p6_software_nodate}}} ist ein Branch namens openthread\_freerots verfügbar. Das Openthread Projekt wurde zuerst auf Basis von FreeROTS entwickelt und ist grundsätzlich voll funktionsfähig. Es gab ein grosses Problem, das bis zum Schluss nicht behoben werden konnte, weshalb das Projekt in die Shared Lib \ref{subsec:SharedLibrary} implementiert werden musste. Das Problem war, dass sich ab ca. 10 Knoten Netzwerkgrösse der SoC nRF52840 nach einiger Zeit in einen Hardfault verfing. Das Problem trat sporadisch und sehr zufällig auf. Es wird vermutet, dass dies auf die RAM Benutzung des SoCs zurückzuführen ist. Laut Google Nest vergrössert sich das RAM mit steigender Anzahl Knoten im Netzwerk. Da das RAM schon zu 50\% mit dem Projekt belegt wird, wäre es möglich, dass dies der Grund ist für den sporadisch auftretenden Hardfault des SoCs.