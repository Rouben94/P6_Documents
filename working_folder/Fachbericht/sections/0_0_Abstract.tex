\begin{abstract}
	Among the most popular low power mesh network protocols in free GHz ISM band the three mesh stacks Bluetooth Mesh, ZigBee and Thread are currently competing against each other. The assignment of this bachelor thesis was to build a consistent test framework for all three mesh-networks to benchmark them under realistic conditions. Due to better compatibility, the nRF52840 SoC from Nordic Semiconductors was the chosen microcontroller for all three network stacks. The benchmark is structured in two parts, a battery powered slave node and a master which is directly connected to a computer. The master node is responsible for controlling the measurement, whereas the slave nodes send benchmark messages to each other. These benchmark messages collect the necessary information to determine latency, RSSI, throughput and active radio time. For a better comparability an apartment house, an apartment and a labor environment were selected as different test benches. The Thread stack results the best in the different test benches. Because of its automatic routing it is able to adapt himself to the environment, as a result the latency of this stack is in every three benches similarly low. Bluetooth Mesh was able to reach the lowest latency with small payload. The ZigBee network stands out with its constant and low latency within one test bench. As a conclusion all of the three networks perform well in case of a home automation. Due to of their own assets and drawbacks it cannot be said this is the best mesh-stack. It depends on the application which mesh network performs the best.
\end{abstract}

