\documentclass[final]{fhnwreport}       %[mode] = draft or final
                                        %{class} = fhnwreport, article, 
                                        %          report, book, beamer, standalone
\input{header}			                %loads all packages, definitions and settings
\addbibresource{literature/bibliography.bib}							
\title{Testumgebung und Performancevergleich von Zigbee, Thread und Bluetooth Mesh Netzwerken}  		        %Project Title
\author{Anklin, Bobst, Horath}      				    %Document Type => Technical Report, ...
\date{\today}          				   %Place and Date

\setcounter{tocdepth}{2} %%%

\begin{document}

%%---TITLEPAGE---------------------------------------------------------------------------------
\thispagestyle{empty}
%	\ohead{\includegraphics[scale=0.5]{Bilder/Logo_FHNW.jpg}}
	\begin{figure}
		 \vspace*{-\topskip}\vspace*{-\headsep}
		\includegraphics[scale=1]{graphics/fhnw_ht_logo_de.pdf}
	\end{figure}
	\begin{center}
		\vspace*{2cm}
		{\huge{\textbf{\thetitle}}}\\
		\vspace*{1cm}
		
		{\huge{Bachelor Thesis}}\\
		\vspace*{0.5cm}
		
		{\scshape\Large Paper - \theauthor \\} \Large{\today}
		\vspace*{3cm}
		\selectlanguage{english}				%ngerman or english
		\thispagestyle{empty}
		\begin{abstract}
			Among the most popular low power mesh network protocols in free GHz ISM band the three mesh stacks Bluetooth Mesh, ZigBee and Thread are currently competing against each other. The assignment of this bachelor thesis was to build a consistent test framework for all three mesh-networks to benchmark them under realistic conditions. Due to better combability, the nRF52840 SoC from Nordic Semiconductors was the chosen microcontroller for all three network stacks. The benchmark is structured in two parts, a battery powered slave node and a master which is directly connected to a computer. The master node is responsible for controlling the measurement, whereas the slave nodes send benchmark messages to each other. These benchmark messages collect the necessary information to determine latency, RSSI, throughput and active radio time. For a better comparability an apartment house, an apartment and a labor environment were selected as different test benches. The Thread stack results the best in the different test benches. Because of its automatic routing it is able to adapt himself to the environment, as a result the latency of this stack is in every three benches similarly low. Bluetooth Mesh was able to reach the lowest latency with small payload. The ZigBee network stands out with its constant and low latency within one test bench. As a conclusion all of the three networks perform well in case of a home automation. Due to of their own assets and drawbacks it cannot be said this is the best mesh-stack. It depends on the application which mesh network performs the best.
		\end{abstract}
	\end{center}
\clearpage

%%---TABLE OF CONTENTS-------------------------------------------------------------------
%\pagenumbering{Roman}		
%\selectlanguage{ngerman}				%ngerman or english
%\tableofcontents
%\clearpage

%%---TEXT--------------------------------------------------------------------------------
\pagenumbering{arabic}

\clearpage

\section{Einleitung}\label{sec:Einleitung}
Im 2.4GHz ISM-Band konkurrieren sich derzeit die drei weit verbreiteten Low Power Mesh Netzwerk Protokolle Bluetooth Mesh, Thread und Zigbee. Alle drei wurden konzipiert für die kabellose Übertragung in sogenannten WSN (Wireless Sensor Networks) oder in Netzen für die Heim Automatisierung. Während Thread und Zigbee den IEEE 802.15.4 Standard als Physical Layer benutzen basiert der BT Mesh Stack auf dem BLE (Bluetooth Low Energy) Standard. Aufgrund der hohen Dichte an Netzwerkprotokollen die das 2.4GHz ISM-Band ebenso nutzen (z.B. Wifi) sind die Störeinflüsse auf die Mesh Protokolle eines der grössten Probleme. Die Protokollstacks begegnen diesem und weiteren Problemen auf unterschiedliche Weise. Diese Unterschiede und schliesslich die Performance der Mesh Netzwerke sollen unter unterschiedlichen Testbedingungen aufgezeigt werden wodurch ein objektiver Vergleich der drei Mesh Protokolle möglich wird. In der nachfolgenden Tabelle ist sind die Hauptmerkmale der drei verschiedenen Mesh-Netzwerke aufgelistet:

\begin{table}[h]
	\centering
	\begin{adjustbox}{width=1\textwidth}
		\begin{tabular}{@{}|l|l|l|l|@{}}
			\toprule
			\multicolumn{4}{|c|}{\textbf{Mesh Netzwerke im Vergleich}}                                                                                                            \\ \midrule
			& \textbf{Bluetooth Mesh}              & \textbf{Thread}                                  & \textbf{ZigBee}                              \\ \midrule
			\textbf{Markt}               & Beleuchtung und Smart Home           & Industrie und Smart Home                         & Beleuchtung, Haus Automation und Messtechnik \\ \midrule
			\textbf{Veröffentlicht}      & 2017                                 & 2015                                             & 2003                                         \\ \midrule
			\textbf{Appllikations Layer} & Mesh Model System                    & Verknüfpbar mit allen IPv6 basierten Protokollen & Cluster Bibliothek                           \\ \midrule
			\textbf{IPv6}                & Nein                                 & Ja                                               & Nein                                         \\ \midrule
			\textbf{Netzwerk Zugriff}    & Smartphone oder Gateway              & Border Router                                    & Gateway                                      \\ \midrule
			\textbf{Ökosysteme}          & Ledvance                             & Google Nest                                      & Ikea, Phillips Hue, Amazon und weitere       \\ \midrule
			\textbf{Routing}             & Managed Flooding                     & Geroutet                                         & Geroutet                                     \\ \midrule
			\textbf{Weiteres}            & Ist direkt mit Smartphone erreichbar & Automatisiertes Verwalten des Netzwerks          & Am meisten verbreitet                        \\ \bottomrule
		\end{tabular}
	\end{adjustbox}
	\caption{Vergleich Mesh Netzwerke}\label{table:VergleichMeshNetzwerk}
\end{table}

Hauptziel ist ein objektiver Vergleich der drei gängigsten Low Power Mesh Netzwerke Bluetooth Mesh, Thread und Zigbee bezüglich deren Leistungsfähigkeit unter wechselnden Bedingungen. Es soll erkennbar werden, welches Protokoll in welchen Bereichen seine Stärken hat und wie es am besten eingesetzt werden kann.


\pagebreak

\clearpage
\section{Methode}\label{sec:Methode}
Um die Performance der drei Mesh Stacks zu vergleichen wurde ein einheitliches Benchmark Konzept erarbeitet. Dieses definiert die Mesh Parameter, Testumgebungen, den Ablauf sowie sämtliche Messgrössen und Messreihen.

\subsection{Messablauf}
Für den Vergleich der 3 Mesh Netzwerkstacks Bluetooth Mesh (BT Mesh), Thread und Zigbee wird ein vom Mesh Protokoll unabhängiges Testkonzept umgesetzt welches in der Abbildung \ref{fig:KonzeptschemaTestablauf} als Konzeptschema dargestellt ist. Die Benchmark Slave Nodes (BSN) in der Abbildung als Sensoren und Aktoren mit unterschiedlichen Funktionalitäten dargestellt, bilden zusammen mit dem Benchmark Master Node (BMN) das zu testende Mesh Netzwerk. Innerhalb des Netzwerks wird dessen Organisation vom jeweiligen Protokoll sichergestellt. Das Testnetzwerk soll ein realitätsnahes Netzwerk nachbilden. Beispielsweise wird eine Hausautomation in einem Einfamilienhaus als Referenz angenommen in welchem jeweils nur gewisse Nodes untereinander Applikationsdaten austauschen. Ein Lichtschalter kommuniziert nur mit einer Lichtquelle und umgekehrt. Der selbe Lichtschalter tauscht jedoch keine Applikationsdaten mit dem Temperatursensor aus. Trotzdem bilden die Nodes zusammen ein Mesh Netzwerk.\\

Die Benchmark Management Station (BMS) welche mit dem BMN via USB/UART kommuniziert, ist zuständig für die Verwaltung und Verarbeitung der Benchmarks. Während eines Benchmark Prozesses sollen sämtliche Messungen jedoch unabhängig von der BMS durchgeführt werden damit allfällige Latenzzeiten der USB/UART Verbindung die Resultate nicht verfälschen.

\begin{figure}[h]
	\centering
	\includegraphics[width=0.55\textwidth]{graphics/Mesh_Testkonzeptschema.png}
	\caption{Konzeptschema Testablauf}
	\label{fig:KonzeptschemaTestablauf}
\end{figure}

\newpage
\subsection{Messaufbau}
Unterschiedliche Testumgebungen sollen die Benchmarks und schlussendlich den Vergleich der 3 Mesh Protokolle aussagekräftiger machen. Nachfolgende Umgebungen mit den entsprechenden Eigenschaften sollen getestet werden. Die Abbildungen zu den Testumgebungen zeigen jeweils die Platzierung der Nodes sowie deren Funktion und Gruppen Zugehörigkeit. Die Farbe Grün identifiziert den Node als Client Node während Blau für einen Server Nodes steht. Die Nummerierung zeigt welcher Node zu welcher Adressgruppe gehört. Ein Client Node in Gruppe 1 sendet jeweils Nachrichten zu allen Server Nodes in der selben Gruppe. 
\newline

\paragraph{Labor}
\begin{wrapfigure}{r}{0.6\textwidth}
	\centering
	\vspace{-20pt}
	\includegraphics[width=0.6\textwidth]{graphics/Testaufbau_Labor.png}
	\caption{Testaufbau Labor}
	\label{fig:TestaufbauLabor}
	\vspace{-10pt}
\end{wrapfigure}
Der Laboraufbau ist ein Extremtest welcher die Leistungsgrenzen der Protokollstacks ausloten soll. Dabei werden die Nodes auf einem Raster gemäss Abbildung \ref{fig:TestaufbauLabor} angeordnet. Die genauen Abmessungen sind der Abbildung zu entnehmen.
\vspace{40mm}

\paragraph{Einfamilienhaus}
\begin{wrapfigure}{r}{0.6\textwidth}
	\centering
	\vspace{-105pt}
	\includegraphics[width=0.6\textwidth]{graphics/Plan_Haus_Raffi.png}
	\caption{Platzierung der Nodes im Einfamilienhaus}\label{fig:Messumgebung2Einfamilienhaus}
	\vspace{-50pt}
\end{wrapfigure}
Die Testgeräte werden in einem Einfamilienhaus installiert und repräsentieren damit eine flächendeckende Heim-Automatisierung. Folgende Eingenschaften soll diese Messung abdecken:

Die Abbildung \ref{fig:Messumgebung2Einfamilienhaus} zeigt den Schnitt des Einfamilienhauses in welchem der Benchmark durchgeführt wurde.

\newpage
\paragraph{Wohnung}
\begin{wrapfigure}{r}{0.6\textwidth}
	\centering
	\vspace{-80pt}
	\includegraphics[width=0.6\textwidth]{graphics/Plan_Wohnung_Cyrill_Nodes_Placement.png}
	\caption{Testaufbau Wohnung}
	\label{fig:TestaufbauWohnung}
	\vspace{-100pt}
\end{wrapfigure}
Ebenfalls als Heim-Automatisierung gedacht werden die Messungen in einer Wohnung durchgeführt.

Bei der Wohnung handelt es sich um eine 3.5 Zimmer Wohnung mit einer Wohnfläche von 122 Quadratmetern. Die genauen Abmessungen sowie die Platzierung der Nodes ist in Abbildung \ref{fig:TestaufbauWohnung} zu sehen.

\vspace{30mm}
\subsection{Vergleichswerte und Messgrössen}\label{subsec:VergleichswerteundMessgrössenMesh}

\subsection{Messerwartung}
Da die Nachrichtenzustellung durch das Routing effizienter realisiert werden kann wird erwartet, dass die beiden auf dem IEEE 802.15.4 Standard aufbauenden Protokolle klare Vorteile haben werden. Hingegen könnte BT Mesh Vorteile haben bei kleiner Belastung des Netzes.

Das menschliche Auge ist in der Lage eine Verzögerung festzustellen, wenn die Latenz vom Knopfdruck bis das Licht angeht mehr als 200ms beträgt. In einer modernen Hausautomation darf natürlich keine Verzögerung wahrgenommen werden. Somit wird erwartet, dass die Latenzzeit im Schnitt unter 200ms bleiben soll. \cite{silicon_laboratories_inc_an1142_2020}
\pagebreak

\clearpage
\section{Ergebnisse}\label{sec:Ergebnisse}
Bei der Durchführung der Mesh Benchmark Messungen wurde für jede Messung unter den entsprechenden Bedingungen ein Messprotokoll resp. eine Auswertung erstellt. Die 8 unterschiedlichen Bedingungen sind in Tabelle \ref{tab:MessungenMeshBenchmark} nochmals zusammengefasst sind.
Diese detaillierten Auswertungen sind im Anhang \ref{app:MessprotokolleMeshBenchmark} dem Bericht angefügt.
Nachfolgend soll exemplarisch eine dieser Auswertungen erläutert werden um aufzuzeigen was diese darstellen und wie diese gelesen werden können.
Eine Interpretation dieser Messresultate ihm Rahmen eines Vergleichs erfolgt schliesslich im Abschnitt \ref{sec:Interpretation}.

\begin{table}[h]
	\centering
	\begin{tabular}{|c|c|c|c|c|c|} 
		\hline
		\textbf{\#}  & \textbf{Msg. Gen.}  & \textbf{Duration}  & \textbf{Msg. Cnt.}  & \textbf{Payload }  & \textbf{Disturbance}  \\ 
		\hline
		1 & Rand & 600s & 60 & Small & No \\ 
		\hline
		2 & Seq & 600s & 60 & Small & No \\ 
		\hline
		3 & Rand & 600s & 60 & Large & No \\ 
		\hline
		4 & Seq & 600s & 60 & Large & No \\ 
		\hline
		5 & Rand & 600s & 600 & Small & No \\ 
		\hline
		6 & Rand & 600s & 60 & Small & Yes \\ 
		\hline
		7 & Seq & 750s & 10 & Small & No \\ 
		\hline
		8 & Seq & 750s & 10 & Large & No \\
		\hline
	\end{tabular}
	\caption{Messungen Mesh Benchmark}
	\label{tab:MessungenMeshBenchmark}
\end{table}


\subsection{Resultate}\label{subsec:Resultate}
Die Messresultate im Anhang \ref{app:MessprotokolleMeshBenchmark} wurden mit den Messindizes 1-8 gemäss Tabelle \ref{tab:MessungenMeshBenchmark}, sowie der entsprechenden Messumgebung (z.B. Wohnung) versehen.
So können die Messungen eindeutig identifiziert werden.
Folgende Einschränkungen müssen dabei jedoch beachtet werden:
Gemäss den Erläuterungen im Abschnitt \ref{subsec:Interpretation} wurden die Messungen 6 - 8 nur im Labor Messaufbau durchgeführt. Von diesen Messungen sind also nur diese Auswertung vorhanden.
Weiter musste bei der Durchführung der Messung 5 festgestellt werden, dass die Resultate unbrauchbar waren.
In der Folge dessen wurde die Auswertung dieser Messung gestrichen. Mehr dazu im Abschnitt  \ref{subsec:Validierung}.


Die Abbildungen \ref{fig:VerteilungderLatenzzeiten} bis \ref{fig:OngoingTransactions} zeigen die Messresultate der Messung 2 in der Messumgebung \textit{Wohnung}. Sie stehen exemplarisch für die Messergebnisse aller Messreihen.
In Abbildung \ref{fig:VerteilungderLatenzzeiten} ist die prozentuale Verteilung der Latenzzeiten pro Hop dargestellt.
In sämtlichen Grafiken werden die Ergebnisse von BT Mesh in blau, jene von Thread in grün und jene von Zigbee in rot dargestellt.
So ist ein direkter Vergleich der Protokolle möglich.
Abbildung \ref{fig:VerteilungderLatenzzeiten} sagt also aus wie viele Nachrichten das Ziel mit einer bestimmten Latenzzeit erreicht haben.
Nachrichten die das Ziel nicht erreicht haben, also Pakete die verloren gegangen sind, werden dabei nicht berücksichtigt.
Im gezeigten Beispiel haben rund 76 Prozent der Nachrichten die im BT Mesh Test versendet wurden das Ziel mit einer maximalen Latenzzeit von 10 Millisekunden erreicht.
Die weitere Verteilung geht bis auf Latenzzeiten von über 300 Millisekunden wobei die Prozentzahl der Nachrichten in diesem Bereich sehr tief ist.
Die Werte für Thread und Zigbee zeigen hingegen eine deutlich schmalere Verteilung der Latenzzeiten.

\begin{figure}[h]
	\centering
	\includegraphics[width=\textwidth]{Latency_2_Wohnung.png}
	\caption{Messung 2 Wohnung: Verteilung der Latenzzeiten pro Hop}
	\label{fig:VerteilungderLatenzzeiten}
\end{figure}

Aus den in Abbildung \ref{fig:VerteilungderLatenzzeiten} aufgezeigten Latenzzeiten wurde der Durchschnitt gebildet und in Abbildung \ref{fig:DurchschnittlicheLatenzzeit} dargestellt.
Es handelt sich dabei wiederum um die Latenzzeit pro Hop. Im Falle von Zigbee ist dies erwähnenswert, da hier die Anzahl Hops nicht ausgelesen werden konnte (siehe Abschnitt \ref{subsubsec:AnzahlHops}) und die Resultate somit mit Vorsicht interpretiert werden müssen. Mehr dazu in der Validierung im Abschnitt \ref{subsec:Validierung}.

Der durchschnittliche Datendurchsatz der mit der Abbildung \ref{fig:DurchschnittlicherDurchsatz} aufgezeigt wird, muss mit der selben Vorsicht betrachtet werden. Denn auch hier werden die Werte pro Hop für die Berechnung verwendet.
Die präsentierten Werte werden errechnet aus der Paketgrösse (Small oder Large) gemäss den Definitionen in Abschnitt \ref{subsec:AllgemeineBenchmarkParameter} und der Latenzzeit des übertragenen Pakets (siehe Gleichung \ref{eq:BerechnungDurchsatz}.
Dabei werden die Werte für den Durchsatz für jedes empfangene Paket berechnet und davon schliesslich der Mittelwert gebildet und in Abbildung \ref{fig:DurchschnittlicherDurchsatz} dargestellt.
Die oben erwähnten Ausreisser bei BT Mesh bewirken nun einen unerwartet hohen Durchsatz bei BT Mesh im Vergleich zu jenem bei Thread welches konstant tiefe Latenzzeiten hat.

\begin{equation}\label{eq:BerechnungDurchsatz}
TP =  \frac{S_{packet} \cdot 8}{t_{lat}}
\end{equation}

\begin{small}
	\begin{center}
		\begin{tabular}{ll}
			$TP$ & Throughput (kBit/s)\\
			$S_{packet}$ & Packetsize (Byte)\\
			$t_{lat}$ & Latency time (ms)\\
		\end{tabular}
	\end{center}
\end{small}

\begin{figure}[!htbp]
	\centering
	\begin{minipage}[b]{0.49\textwidth}
		\centering
		\includegraphics[width=\textwidth]{Average_Latency_per_Hop.png}
		\caption{Messung 2 Wohnung: Durchschnittliche Latenzzeit pro Hop}
		\label{fig:DurchschnittlicheLatenzzeit}
	\end{minipage}
	\begin{minipage}[b]{0.49\textwidth}
		\centering
		\includegraphics[width=\textwidth]{Average_Throughput_per_Hop.png}
		\caption{Messung 2 Wohnung: Durchschnittlicher Durchsatz pro Hop}
		\label{fig:DurchschnittlicherDurchsatz}
	\end{minipage}
\end{figure}

In der Abbildung \ref{fig:DurchschnittlicherPaketverlust} wird der prozentuale Paketverlust gezeigt über die gesamte Anzahl Nachrichten die während dem Benchmark versendet wurden.
Die Paketverluste von einzelnen Client-Server Beziehungen werden nicht separat ausgewertet.
Wiederum im Beispiel von BT Mesh sind in dieser Messung 2.07 \% der Pakete nicht am Ziel angekommen.

\begin{figure}[!htbp]
	\centering
	\begin{minipage}[b]{0.49\textwidth}
		\centering
		\includegraphics[width=\textwidth]{Average_Packet_Loss.png}
		\caption{Messung 2 Wohnung: Durchschnittlicher Paketverlust}
		\label{fig:DurchschnittlicherPaketverlust}
	\end{minipage}
	\begin{minipage}[b]{0.49\textwidth}
		\centering
		\includegraphics[width=\textwidth]{Average_Radio_Energy_Consumption.png}
		\caption{Messung 2 Wohnung: Durchschnittlicher Energieverbrauch}
		\label{fig:DurchschnittlicherEnergieverbrauch}
	\end{minipage}
\end{figure}

Mit dem Diagramm in Abbildung \ref{fig:DurchschnittlicherEnergieverbrauch} wird schliesslich noch der durchschnittliche Energieverbrauch der Protokolle dargestellt.
Dieser ist abgeleitet aus der \textit{Active Radio Time} (siehe Abschnitt \ref{subsubsec:Vergleichswerte}) welche direkt auf dem nRF52840 SoC mit der entsprechenden API ausgelesen werden kann.
Die \textit{Active Radio Time} wurde schliesslich mit dem Strombedarf des SoC's bei definierter Speisesannung von 3V verrechnet.
Gemäss den Angaben in der Tabelle \ref{tab:EigenschaftennRF52840SoC} aus Abschnitt \ref{subsec:SystemonChip} beträgt der Strombedarf bei aktivem Funkmodul im Sendemodus 4.8mA.
Eine solche Berechnung erlaubt einen qualitativen Vergleich des Energiebedarf unter den 3 Protokollen da die Umsetzung auf der gleichen Hardware erfolgte.
Die Werte in der Abbildung \ref{fig:DurchschnittlicherEnergieverbrauch} sind jedoch keine quantitativen Verbrauchswerte und können deshalb nur im Kontext des Vergleichs verwendet werden.
Der Strombedarf sämtlicher Peripherie wurde nicht berücksichtigt da dieser prinzipiell bei allen Protokollen identisch ist.

\begin{figure}[h]
	\centering
	\includegraphics[width=\textwidth]{Ongoing_Transactions_and_Latency.png}
	\caption{Messung 2 Wohnung: Ongoing Transactions und Verlauf der Latenzzeiten über die Messdauer.}
	\label{fig:OngoingTransactions}
\end{figure}

Die letzte Grafik gemäss Abbildung \ref{fig:OngoingTransactions} zeigt den Verlauf der \textit{Ongoing Transactions} sowie der Latenzzeiten über die Gesamtdauer einer Messung.
In diesem Fall beträgt die Dauer 600 Sekunden.
Die Grafik soll einen Eindruck darüber vermitteln, wie die Stacks damit umgehen, wenn vielen Nachrichten zur selben Zeit versendet werden.
Die \textit{Ongoing Transactions} welche als durchgezogene Linien dargestellt sind, zeigen zu welchem Zeitpunkt wie viele Nachrichten in der Übermittlung sind.
Da die Nachrichten in diesem Beispiel sequentiellen versendet werden gibt es nur sehr geringe Ausschläge welche im unteren Bildrand zu sehen sind.
Die logarithmische Darstellung der Latenzzeiten als gestrichelte Linie bestätigt die Beobachtungen die in der Abbildung \ref{fig:VerteilungderLatenzzeiten} bereits gemacht wurden.
Zigbee wie auch Thread weisen einen ziemlich regelmässigen Verlauf der Latenzzeiten auf.
BT Mesh hingegen zeigt starke Ausreisser.
\pagebreak

\input{sections/4_Interpretation}
\pagebreak

\clearpage
\section{Validierung und Verifizierung}\label{sec:ValidierungVerifizierung}

\subsection{Validierung}\label{subsec:Validierung}
Die durchgeführten Messungen haben aussagekräftige Resultate geliefert welche jedoch stark vom gewählten Vorgehen abhängig sind. Dieses Vorgehen soll nun kritisch überprüft und allfällige Mängel im Konzept sowie der Umsetzung aufgezeigt werden. Zudem werden systematische Messfehler deklariert.

\paragraph{Large Payload}
Die Resultate der Messreihen 3, 4 und 8, welche mit einer grossen Payload durchgeführt wurden, sind nur für Thread und Zigbee aussagekräftig. Der BT Mesh Stack liefert bei diesen Messungen keine brauchbaren Resultate. Eine Recherche zu diesem Problem hat ergeben, dass durch die Fragmentierung einer 32 Byte Payload die sichere und schnelle Übertragung der Daten nicht mehr gewährleistet werden kann.

\paragraph{Zigbee Latency}
Wie bereits erwähnt, konnte bei Zigbee die Anzahl Hops die ein Paket passiert hat nicht ausgewertet werden.
Der Forumsbeitrag \href{https://devzone.nordicsemi.com/f/nordic-q-a/63815/zigbee---read-number-of-hops-radius}{\textit{Zigbee - Read number of hops (radius)\footnote{\url{https://devzone.nordicsemi.com/f/nordic-q-a/63815/zigbee---read-number-of-hops-radius}}}} bestätigt, dass der entsprechende Wert mit der verwendeten SDK nicht ausgelesen werden kann.
Als Folge dessen kann schliesslich die Latenzzeit nur als Total über die gesamte Strecke erfasst werden. In der Auswertung verschafft dies BT Mesh und Thread fälschlicherweise einen Vorteil gegenüber Zigbee.
Die Auswertung der totalen Latenzzeit bei allen 3 Protokollen könnte das Problem lösen.
Dies würde jedoch dem Messkonzept widersprechen und wurde deshalb nicht für alle Messungen umgesetzt.
Die Abbildungen \ref{fig:DurchschnittlicheLatenzzeitValidierung} und \ref{fig:DurchschnittlicheLatenzzeitohneHopsValidierung} zeigen den Unterschied am Beispiel der oben analysierten Messung \ref{sec:Ergebnisse}.
Während links die Latenzzeit pro Hop dargestellt ist, zeigt die rechte Grafik die totale Latenzzeit.
Der Unterschied ist vorallem bei Thread deutlich erkennbar.
Bereits in der Abbildung \ref{fig:VerteilungderLatenzzeiten} ist das Problem zu erkennen.
Die Statistik der Latenzzeiten von Zigbee hat zwei Hauptsäulen bei 40ms und 70ms was auf einen Hop hindeutet.


\begin{figure}[!htbp]
	\centering
	\begin{minipage}[b]{0.49\textwidth}
		\centering
		\includegraphics[width=\textwidth]{Average_Latency_per_Hop.png}
		\caption{Durchschnittliche Latenzzeit pro Hop}
		\label{fig:DurchschnittlicheLatenzzeitValidierung}
	\end{minipage}
	\begin{minipage}[b]{0.49\textwidth}
		\centering
		\includegraphics[width=\textwidth]{Average_Latency_without_Hops.png}
		\caption{Durchschnittliche Latenzzeit ohne Berücksichtigung der Hops.}	\label{fig:DurchschnittlicheLatenzzeitohneHopsValidierung}
	\end{minipage}
\end{figure}

\newpage
\paragraph{Nachrichten Dichte}
Bei der Definition der Messreihen \ref{sec:Interpretation} wurden zu Beginn nur die Messungen 1 bis 6 spezifiziert.
Die Messreihen 7 und 8 kamen erst nachträglich hinzu als festgestellt werden musste, dass die Dichte der Nachrichten für den BT Mesh Stack zu hoch gewählt wurde.
Dieser schien überfordert besonders bei zufälliger Nachrichten Generierung.
Thread und Zigbee zeigten indes keine Mühe mit der Dichte von 60 Nachrichten in 600 Sekunden pro Node.
Die Resultate der Messungen 7 und 8 haben schliesslich gezeigt, dass die Reduktion der Nachrichtendichte einen positiven Einfluss hat.

\paragraph{Group addressing mode}
Der \textit{Group addressing mode} wurde für die 3 Protokolle unterschiedlich definiert.
Erste Tests vor den eigentlichen Benchmarks haben gezeigt, dass eine Multicast Adressierung bei Zigbee unbrauchbar ist.
Deshalb hat man sich entschieden bei Zigbee eine Unicast Adressierung umzusetzen.
Während den Benchmarks musste schliesslich festgestellt werden, dass Zigbee durch diese Änderung ein deutlicher Vorteil erlangt hat.
Besonders auf die Paketverlustrate hat sich die Unicast Adressierung positiv ausgewirkt denn Unicast Nachrichten werden im IEEE 802.15.4 Standard auf MAC Ebene quittiert.
Multicast resp. Broadcast hingegen nicht.

Auch in der Messung Nr. 5 hätte sich die Unicast Adressierung für Zigbee positiv ausgewirkt da die Netzbelastung deutlich geringer gewesen wäre.

\paragraph{Durchschnittswerte in den Resultaten}
In den Ergebnissen \ref{sec:Ergebnisse} wurden sämtliche Durchschnittswerte als Mittelwerte einschliesslich allfälliger Ausreisser aus den Messwerten gebildet.
Dadurch wurden gewisse Resultate deutlich verfälscht.
In einer neuen Auswertung müsste die Ursache für die einzelnen Ausreisser genauer geklärt werden und diese allenfalls gestrichen werden.


\subsection{Verifizierung}\label{subsec:Verifizierung}
Eine Verifizierung der Messresultate konnte nur anhand des Referenzberichts \textit{AN1142: Mesh Network Performance
	Comparison\footnote{\url{https://www.silabs.com/documents/public/application\%2Dnotes/an1142\%2Dmesh\%2Dnetwork\%2Dperformance\%2Dcomparison.pdf} \cite{silicon_laboratories_inc_an1142_2020}}} von \textit{Silicon Labs} gemacht werden.
Dieser ist auf der Webseite von \textit{Silicon Labs} einsehbar.

Der Vergleich der Messergebnisse hat gezeigt, dass die Grössenordnung der Resultate mit jenen aus dem Bericht von Silicon Labs übereinstimmt.
Selbst die Ausreisser in der Latenzzeit bei BT Mesh liegen im selben Rahmen.
Zudem kann die klare Abhängigkeit der Resultate von der Grösse der Payload bestätigt werden.
Einige Unterschiede sind jedoch in der Verteilung der Latenzzeiten erkennbar.
Im Referenzbericht ist diese in einem Bereich zwischen 20ms und 60ms ziemlich regelmässig während in den Ergebnissen dieser Thesis die Verteilung unregelmässiger daherkommt.
\pagebreak

\clearpage
%%---BIBLIOGRAPHY------------------------------------------------------------------------
{\sloppypar
\printbibliography[heading=bibintoc]
\label{sec:lit}
%\selectlanguage{ngerman}				%ngerman or english
%\printbibliography
}

%%---List of Figures------------------------------------------------------------------------
%\listoffigures

%%---APPENDIX----------------------------------------------------------------------------
%\begin{appendix} 

\addcontentsline{toc}{section}{Anhang}


%**********************Aufgabenstellung***************************
%\includepdf[pages={1}, nup=1x1, landscape=false, scale=0.9 ,offset=0 -45, pagecommand={\section{Aufgabenstellung}\label{app:Aufgabenstellung}\thispagestyle{myheadings}}]{appendix/P6_Aufgabenstellung_Wireless_Controller_for_Smart_Systems.pdf}
%
%\includepdf[pages={2-5}, nup=1x1, landscape=false, scale=0.9 ,offset=0 -45, pagecommand={\thispagestyle{myheadings}}]{appendix/P6_Aufgabenstellung_Wireless_Controller_for_Smart_Systems.pdf}
%
%
%%**********************Pflichtenheft***************************
%\includepdf[pages={1}, nup=1x1, landscape=false, scale=0.95 ,offset=0 -45, pagecommand={\section{Pflichtenheft}\label{app:Pflichtenheft}\thispagestyle{myheadings}}]{appendix/P6_Pflichtenheft.pdf}
%
%\includepdf[pages={2-19}, nup=1x1, landscape=false, scale=0.95 ,offset=0 -45, pagecommand={\thispagestyle{myheadings}}]{appendix/P6_Pflichtenheft.pdf}
%
%%***************EMV Bericht Abstrahlung Antennen*********************
%\includepdf[pages={1}, nup=1x1, landscape=false, scale=0.95 ,offset=0 -45, pagecommand={\section{Bericht emv Messung Development Kits}\label{app:BerichtemvMessungDevelopmentKits}\thispagestyle{myheadings}}]{appendix/emv_Bericht_FS20.pdf}
%
%\includepdf[pages={2-13}, nup=1x1, landscape=false, scale=0.95 ,offset=0 0, pagecommand={\thispagestyle{myheadings}}]{appendix/emv_Bericht_FS20.pdf}
%
%%***************Messprotokolle Mesh Benchmark*********************
%\includepdf[pages={1}, nup=1x1, landscape=false, scale=0.95 ,offset=0 -20, pagecommand={\section{Messprotokolle Mesh Benchmark}\label{app:MessprotokolleMeshBenchmark}\thispagestyle{myheadings}}]{appendix/Messprotokolle_Labor.pdf}
%
%\includepdf[pages={2-14}, nup=1x1, landscape=false, scale=0.95 ,offset=0 0, pagecommand={\thispagestyle{myheadings}}]{appendix/Messprotokolle_Labor.pdf}
%
%\includepdf[pages={1}, nup=1x1, landscape=false, scale=0.95 ,offset=0 0, pagecommand={\thispagestyle{myheadings}}]{appendix/Messprotokolle_Haus.pdf}
%
%\includepdf[pages={2-8}, nup=1x1, landscape=false, scale=0.95 ,offset=0 0, pagecommand={\thispagestyle{myheadings}}]{appendix/Messprotokolle_Haus.pdf}
%
%\includepdf[pages={1}, nup=1x1, landscape=false, scale=0.95 ,offset=0 0, pagecommand={\thispagestyle{myheadings}}]{appendix/Messprotokolle_Wohnung.pdf}
%
%\includepdf[pages={2-8}, nup=1x1, landscape=false, scale=0.95 ,offset=0 0, pagecommand={\thispagestyle{myheadings}}]{appendix/Messprotokolle_Wohnung.pdf}
%
%%****************************Paper*************************
%\includepdf[pages={1}, nup=1x1, landscape=false, scale=0.95 ,offset=0 -45, pagecommand={\section{Paper}\label{app:Paper}\thispagestyle{myheadings}}]{appendix/Paper.pdf}
%
%\includepdf[pages={2-8}, nup=1x1, landscape=false, scale=0.95 ,offset=0 0, pagecommand={\thispagestyle{myheadings}}]{appendix/Paper.pdf}

%***************Random Value Generation*********************
\includepdf[pages={1}, nup=1x1, landscape=false, scale=0.8 ,offset=0 -20, pagecommand={\section{Random Traffic Generation}\label{app:RandomTrafficGeneration}\thispagestyle{myheadings}}]{appendix/RANDOM.ORG_Integer_Set_Generator.pdf}


\end{appendix}



%%---NOTES for DEBUG---------------------------------------------------------------------
\ifdraft{%Do this only if mode=draft
%%requires \usepackage{todonotes})
\newpage
\listoftodos[\section{Todo-Notes}]
\clearpage
}
{%Do this only if mode=final
}

\end{document}
