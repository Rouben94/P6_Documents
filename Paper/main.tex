%%%%%%%%%%%%%%%%%%%%%%%%%%%%%%%%%%%%%%%%%%%%%%%%%%%%%%%
% A template for Wiley article submissions developed by 
% Overleaf for the Overleaf-Wiley pilot which ran 
% during 2017 and 2018.
% 
% This template is no longer supported, but is provided
% for historical reference. Last updated January 2019.
%
% Please note that whilst this template provides a 
% preview of the typeset manuscript for submission, it 
% will not necessarily be the final publication layout.
%
% Document class options:
% =======================
% blind: Anonymise all author, affiliation, correspondence
%        and funding information.
%
% lineno: Adds line numbers.
%
% serif: Sets the body font to be serif. 
%
% twocolumn: Sets the body text in two-column layout. 
% 
% num-refs: Uses numerical citation and references style 
%           (Vancouver-authoryear).
%
% alpha-refs: Uses author-year citation and references style
%             (rss).
%
% Using other bibliography styles:
% =======================
%
% To specify a different bibiography style
%
% 1) Do not use either num-refs or alpha-refs in documentclass.
% 2) Load natbib package with the options set as needed.
% 3) Use the \bibliographystyle command to specify the style
% 
% Included NJD styles are: 
%   WileyNJD-ACS
%   WileyNJD-AMA
%   WileyNJD-AMS
%   WileyNJD-APA
%   WileyNJD-Harvard
%   WileyNJD-VANCOUVER
%
% or you may upload an alternative .bst file 
% (if requested by the journal).
%
% Examples:
% =======================
%% Example: Using numerical, sort-by-authors citations.
\documentclass[num-refs, serif]{wiley-article}

%% Example: Using author-year citations and anonymising submission
% \documentclass[blind,alpha-refs]{wiley-article}

%% Example: Using unsrtnat for numerical, in-sequence citations
% \documentclass{wiley-article}
% \usepackage[numbers]{natbib}
% \bibliographystyle{unsrtnat}

%% Example: Using WileyNJD-AMA reference style and superscript
%%          citations, two-column and serif fonts for AIChE
% \documentclass[serif,twocolumn,lineno]{wiley-article}
% \usepackage[super]{natbib}
% \bibliographystyle{WileyNJD-AMA}
% \makeatletter
% \renewcommand{\@biblabel}[1]{#1.}
% \makeatother

% Add additional packages here if required
\usepackage{siunitx}

\usepackage[textsize=footnotesize]{todonotes}

% Update article type if known
\papertype{FHNW Paper}

\title{Perfomancevergleich von Zigbee, Thread und Bluetooth Mesh Netzwerken}

% List abbreviations here, if any. Please note that it is preferred that abbreviations be defined at the first instance they appear in the text, rather than creating an abbreviations list.
\abbrevs{ABC, a black cat; DEF, doesn't ever fret; GHI, goes home immediately.}

% Include full author names and degrees, when required by the journal.
% Use the \authfn to add symbols for additional footnotes and present addresses, if any. Usually start with 1 for notes about author contributions; then continuing with 2 etc if any author has a different present address.
\author[1]{Cyrill Horath}
\author[1]{Raffael Anklin}
\author[1]{Robin Bobst}

% Include full affiliation details for all authors
\affil[1]{Institut für ??, Fachhochschule Nordwestschweiz, Windisch, Aargau, 5210, Schweiz}

\corraddress{Team Blau, Institut für ??, Fachhochschule Nordwestschweiz, Windisch, Aargau, 5210, Schweiz}
\corremail{TeamBlau@email.com}

%\presentadd[\authfn{2}]{Department, Institution, City, State or Province, Postal Code, Country}

%\fundinginfo{Funder One, Funder One Department, Grant/Award Number: 123456, 123457 and 123458; Funder Two, Funder Two Department, Grant/Award Number: 123459}

% Include the name of the author that should appear in the running header
\runningauthor{Perfomancevergleich Zigbee, Thread, Bluetooth}

\begin{document}
\begin{figure}
	\includegraphics[scale=1]{graphics/fhnw_ht_logo_de.pdf}
\end{figure}

\begin{frontmatter}
\maketitle

\begin{abstract}
\todo[inline]{Abstract hinzufügen}

% Please include a maximum of seven keywords
\keywords{keyword 1, \emph{keyword 2}, keyword 3, keyword 4, keyword 5, keyword 6, keyword 7}
\end{abstract}
\end{frontmatter}

\section{Einleitung}
\todo[inline]{In der Einleitung sollen die drei verschiedenen Stacks kurz und knapp erläutert werden und welche Vor- und Nachteile diese haben.}

\subsection{Bluetooth Mesh}
\todo[inline]{Kurze Beschreibung vom Stack}

\subsection{OpenThread}
\todo[inline]{Kurze Beschreibung vom Stack}

\subsection{ZigBee}
\todo[inline]{Kurze Beschreibung vom Stack}

\section{Methode}
\todo[inline]{Hier sollen die Messmethoden dargelegt werden. (Wie wurde gemessen, Programmablauf, Genauigkeit, Wie wurden die Messungen aufgezeichnet und gespeichert usw.)}
\todo[inline]{Die drei Stacks müssen vielleicht nicht unterteilt werden.}
\subsection{Bluetooth Mesh}
\todo[inline]{Messmethode Bluetooth Mesh}

\subsection{OpenThread}
\todo[inline]{Messmethode OpenThread}

\subsection{ZigBee}
\todo[inline]{Messmethode ZigBee}

\section{Ergebnisse}
\todo[inline]{Die Ergebnisse sollen hier nach verschiedenen Kriterien dargestellt werden (Anzahl Nodes, Anzahl Hops, usw.)}

\section{Interpretation}
\todo[inline]{Interpretation der Ergebnisse (Was fällt besonders auf, wo sind die stärken und schwächen der einzelnen Netzwerke, usw.)}

%\begin{figure}[bt]
%\centering
%\includegraphics[width=6cm]{Graphics/example-image-rectangle}
%\caption{Testbild}
%\end{figure}

%\begin{table}[bt]
%\caption{Test Tabelle}
%\begin{threeparttable}
%\begin{tabular}{lccrr}
%\headrow
%\thead{Variables} & \thead{JKL ($\boldsymbol{n=30}$)} & \thead{Control ($\boldsymbol{n=40}$)} & \thead{MN} & \thead{$\boldsymbol t$ (68)}\\
%Age at testing & 38 & 58 & 504.48 & 58 ms\\
%Age at testing & 38 & 58 & 504.48 & 58 ms\\
%Age at testing & 38 & 58 & 504.48 & 58 ms\\
%Age at testing & 38 & 58 & 504.48 & 58 ms\\
%\hiderowcolors
%stop alternating row colors from here onwards\\
%Age at testing & 38 & 58 & 504.48 & 58 ms\\
%Age at testing & 38 & 58 & 504.48 & 58 ms\\
%\hline  % Please only put a hline at the end of the table
%\end{tabular}
%
%\begin{tablenotes}
%\item JKL, just keep laughing; MN, merry noise.
%\end{tablenotes}
%\end{threeparttable}
%\end{table}

\section*{Ergänzende Informationen}
\todo[inline]{Infos die evtl. wichtig sind aber nicht unbedingt in den Kontext gehören}

% Submissions are not required to reflect the precise reference formatting of the journal (use of italics, bold etc.), however it is important that all key elements of each reference are included.
\bibliography{bibliography}

\end{document}